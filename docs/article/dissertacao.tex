% ---------------------------------------------------------------
% Modelo LaTex para dissertação e tese do programa de Pós Graduação em Ciência da Computação da UFABC
% ---------------------------------------------------------------

\documentclass[
	% -- opções da classe memoir --
	12pt,					% tamanho da fonte
	openright,				% capítulos começam em pág ímpar (insere página vazia caso preciso)
	oneside,					% para impressão em verso e anverso. Oposto a oneside
	a4paper,					% tamanho do papel. 
	% -- opções da classe abntex2 --
	%chapter=TITLE,			% títulos de capítulos convertidos em letras maiúsculas
	%section=TITLE,			% títulos de seções convertidos em letras maiúsculas
	%subsection=TITLE,		% títulos de subseções convertidos em letras maiúsculas
	%subsubsection=TITLE,	% títulos de subsubseções convertidos em letras maiúsculas
	% -- opções do pacote babel --
	english,					% idioma adicional para hifenização
	%french,					% idioma adicional para hifenização
	%spanish,				% idioma adicional para hifenização
	brazilian			   % o último idioma é o principal do documento
	]{abntex2}

% ---------------------
% Pacotes OBRIGATÓRIOS
% ---------------------
\usepackage{lmodern}				% Usa a fonte Latin Modern			
\usepackage[T1]{fontenc}			% Selecao de codigos de fonte.
\usepackage[utf8]{inputenc}		% Codificacao do documento (conversão automática dos acentos)
\usepackage{lastpage}			% Usado pela Ficha catalográfica
\usepackage{indentfirst}			% Indenta o primeiro parágrafo de cada seção.
\usepackage{color}				% Controle das cores
\usepackage[dvipsnames]{xcolor,colortbl}
\usepackage{graphicx,graphicx}	% Inclusão de gráficos
\usepackage{epsfig,subfig}		% Inclusão de figuras
\usepackage{microtype} 			% Melhorias de justificação
% ---------------------
		
% ---------------------
% Pacotes ADICIONAIS
% ---------------------
\usepackage{lipsum}						% Geração de dummy text
\usepackage{amsmath,amssymb,mathrsfs}	% Comandos matemáticos avançados 
\usepackage{setspace}  					% Para permitir espaçamento simples, 1 1/2 e duplo
\usepackage{verbatim}					% Para poder usar o ambiente "comment"
\usepackage{tabularx} 					% Para poder ter tabelas com colunas de largura auto-ajustável
\usepackage{afterpage} 					% Para executar um comando depois do fim da página corrente
\usepackage{url} 						% Para formatar URLs (endereços da Web)
\usepackage{float} 						% Pacote para formatação de imagens
\usepackage{listings}                   % Pacote para formatação de código
\usepackage{plantuml}                   % Formatação de Arquivos em PlantUML
\usepackage{comment}                    % Poder comentar em bloco
\usepackage{placeins}
\usepackage[nameinlink]{cleveref}
\usepackage{todonotes}
\usepackage{threeparttable}
\usepackage{booktabs, multirow, rotating}

% ---------------------
% Formatação de código Java
% ---------------------
\definecolor{dkgreen}{rgb}{0,0.6,0}
\definecolor{gray}{rgb}{0.5,0.5,0.5}
\definecolor{mauve}{rgb}{0.58,0,0.82}

\lstset{
    frame=tb,
    aboveskip=3mm,
    belowskip=3mm,
    showstringspaces=false,
    columns=flexible,
    basicstyle={\small\ttfamily},
    numbers=left,
    numberstyle=\tiny\color{gray},
    keywordstyle=\color{blue},
    commentstyle=\color{dkgreen},
    stringstyle=\color{mauve},
    breaklines=true,
    breakatwhitespace=true,
    tabsize=4
}

% --------------------------
% Formatação de código em Kotlin
% Reference: https://github.com/cansik/kotlin-latex-listing
% --------------------------
\lstdefinelanguage{Kotlin}{
  comment=[l]{//},
  commentstyle={\color{gray}\ttfamily},
  emph={filter, first, firstOrNull, forEach, lazy, map, mapNotNull, println},
  emphstyle={\color{OrangeRed}},
  identifierstyle=\color{black},
  keywords={!in, !is, abstract, actual, annotation, as, as?, break, by, catch, class, companion, const, constructor, continue, crossinline, data, delegate, do, dynamic, else, enum, expect, external, false, field, file, final, finally, for, fun, get, if, import, in, infix, init, inline, inner, interface, internal, is, lateinit, noinline, null, object, open, operator, out, override, package, param, private, property, protected, public, receiveris, reified, return, return@, sealed, set, setparam, super, suspend, tailrec, this, throw, true, try, typealias, typeof, val, var, vararg, when, where, while},
  keywordstyle={\color{NavyBlue}\bfseries},
  morecomment=[s]{/*}{*/},
  morestring=[b]",
  morestring=[s]{"""*}{*"""},
  ndkeywords={@Deprecated, @JvmField, @JvmName, @JvmOverloads, @JvmStatic, @JvmSynthetic, Array, Byte, Double, Float, Int, Integer, Iterable, Long, Runnable, Short, String, Any, Unit, Nothing},
  ndkeywordstyle={\color{BurntOrange}\bfseries},
  sensitive=true,
  stringstyle={\color{ForestGreen}\ttfamily},
}

% --------------------------
% Formatação de arquivos .toml
% --------------------------
\lstdefinelanguage{toml}{
    comment = [l]{\#},
    keywords={libraries, versions, plugins, bundles},
    ndkeywords={module, group, name, version, ref},
    keywordstyle={\color{BurntOrange}\bfseries},
    ndkeywordstyle={\color{BurntOrange}},
    keywords = {true, false},
    morestring = [b]",
    stringstyle={\color{ForestGreen}\ttfamily},
}

% ---------------------
% Pacotes de CITAÇÕES
% ---------------------
\usepackage[english,hyperpageref]{backref}	% Paginas com as citações na bibl
\usepackage[alf]{abntex2cite}				% Citações padrão ABNT (alfa)
%\usepackage[num]{abntex2cite}				% Citações padrão ABNT (numericas)

% ---------------------

% Configurações de CITAÇÕES para abntex2
\include{extras/conf_citacoes}

% Inclusão de dados para CAPA e FOLHA DE ROSTO (título, autor, orientador, etc.)
% ---
% Informações de dados para CAPA e FOLHA DE ROSTO
% ---
\titulo{Classificação e análise de defeitos comuns em softwares e como mitigá-los utilizando diferentes práticas de desenvolvimento}
\autor{Vinícius de Oliveira Campos dos Reis}
\local{Santo André - SP}
\data{Dezembro de 2023}
\orientador{Emílio Francesquini}
\instituicao{%
  Universidade Federal do ABC -- UFABC
  \par
  
  Centro de Matemática, Computação e Cognição
  \par
  Bacharelado em Ciência da Computação}
\tipotrabalho{Projeto de Graduação}
% O preambulo deve conter o tipo do trabalho, o objetivo,
% o nome da instituição e a área de concentração
\preambulo{Projeto de Gradulação em Computação III (PGC III) enviado para o Bacharelado em Ciência da Computação, como parte dos requisitos mínimos para obter o diploma do Bacharelado em Ciência da Computação.}
% ---

% Inclui Configurações de aparência do PDF Final
\include{extras/conf_pdf}

% O tamanho da identação do parágrafo é dado por:
\setlength{\parindent}{1.3cm}

% Controle do espaçamento entre um parágrafo e outro:
\setlength{\parskip}{0.2cm}  % tente também \onelineskip
\setlength{\marginparwidth}{2cm}

% ---------------------
% Compila o indice
% ---------------------
\makeindex
% ---------------------

%%%%%%%%%%%%%%%%%%%%%%%%%%%
%%  INICIO DO DOCUMENTO  %%
%%%%%%%%%%%%%%%%%%%%%%%%%%%
\begin{document}
\setlength{\marginparwidth}{2cm}
%\renewcommand*\contentsname{Summary}
%\renewcommand{\bibname}{References}
%\renewcommand{\figurename}{Figure}
%\renewcommand{\apendicename}{APPENDIX}
%\renewcommand{\apendicesname}{Appendix}
\selectlanguage{brazilian}


% Retira espaço extra obsoleto entre as frases.
\frenchspacing

% ----------------------------------------------------------
% ELEMENTOS PRÉ-TEXTUAIS (Capa, Resumo, Abstract, etc.)
% ----------------------------------------------------------
\pretextual

% Capa
\include{pretextual/capa}

% Folha de rosto (o * indica que haverá a ficha bibliográfica)
\imprimirfolhaderosto

% Imprimir Ficha Catalografica
%\include{pretextual/catalografica}

% Inserir Folha de Aprovação
%\include{pretextual/assinaturas}

% Dedicatória
%\include{pretextual/dedicatoria}

% Agradecimentos
%% ---
% Agradecimentos
% ---
\begin{agradecimentos}


Agradeço ao meu orientador, XXXXXXXXX, por todos os conselhos, pela paciência e ajuda nesse período.

Aos meus amigos ...

Aos professores ...

À XXXXXX pelo apoio financeiro para realização deste trabalho de pesquisa.

\end{agradecimentos}
%% ---

% Epígrafe
%\include{pretextual/epigrafe}

% Resumo e Abstract
% ---
% RESUMOS
% ---

% RESUMO em português
\begin{resumo}
Com a modernização da sociedade, seu funcionamento está cada vez mais vinculado a execução de inúmeros tipos e aplicações de softwares em todos seus segmentos. O processo de desenvolvimento destas aplicações está sujeito a falhas. Falhas que podem gerar problemas que afetem sua plena atividade. Por conta disso, é importante adotarmos as melhores práticas que garantam a plena e correta execução destas aplicações. O objetivo principal deste trabalho é analisar quais são os defeitos mais frequentes em softwares e se estão relacionados com a pilha de tecnologias utilizada para sua criação. Iniciamos com uma revisão estruturada da literatura científica, em busca de uma classificação consolidada de defeitos de software. Com esta classificação, utilizando uma base de ocorrências de defeitos capturados numa aplicação real amplamente publicada, a aplicamos e analisamos os defeitos mais comuns visando entender se existem práticas de desenvolvimento, dentro da própria pilha utilizada, que possam mitigar estas ocorrências. Exploramos também ferramentas externas ao código-fonte do software em questão que seja capaz de analisá-lo em busca de vulnerabilidades. Como exemplo, utilizamos uma ferramenta de \textit{lint} chamada \textit{detekt} e registramos o processo de implementação, execução e revisão dos resultados obtidos em outro projeto de exemplo.

\vspace{\onelineskip}
 
\noindent 
\textbf{Palavras-chaves}: Java. Kotlin. Android. engenharia de software. odc. classificação. defeito. bug. linter. lint. detekt.
\end{resumo}

% ABSTRACT in english
\begin{resumo}[Abstract]
 \begin{otherlanguage*}{english}

With the modernization of society, its operation is increasingly tied to the execution of countless types and applications of software in all its segments. The development process of these applications is prone to failures. Failures that might generate problems that affect its full functioning. Because of this, it is important that we adopt the best practices that guarantee the full and correct execution of these applications. The main goal of this work is to analyze which are the most frequent defects in software and whether they are related to the stack used to create it. We started with a systematic review of the scientific literature, searching for a consolidated classification of software defects. With this classification, using a database of crash occurrences captured in a widely published real app, we classified and analyze the most common defects aiming to understand whether there are development practices, within the own stack used, that can mitigate these occurrences. We also explored external tools from the software's source code that are capable of analyzing it in search of vulnerabilities. As an example, we used a lint tool called \textit{detekt} and documented the process of implementation, execution and review of the results in another example project.

\vspace{\onelineskip}
 
\noindent 
\textbf{Keywords}: Java. Kotlin. Android. software engeneering. odc. classification. bug. linter. lint. detekt.
 \end{otherlanguage*}
\end{resumo}

% Lista de ilustrações
%\pdfbookmark[0]{\listfigurename}{lof}
%\listoffigures*
%\cleardoublepage

% Lista de tabelas
%\pdfbookmark[0]{\listtablename}{lot}
%\listoftables*
%\cleardoublepage

% Lista de abreviaturas e siglas
%\begin{siglas}
%  \item[ABNT] Associação Brasileira de Normas Técnicas
 % \item[abnTeX] Normas para TeX
%  \item [DOM] Domain Object Model
%  \item [PGC] Programa de Graduação em Computação
%\end{siglas}

% Lista de símbolos
% \begin{simbolos}
%   \item[$ \Gamma $] Letra grega Gama
%  \item[$ \Lambda $] Lambda
%  \item[$ \zeta $] Letra grega minúscula zeta
%  \item[$ \in $] Pertence
% \end{simbolos}

% Inserir o Summary
\pdfbookmark[0]{\contentsname}{Summary}
\tableofcontents*
\cleardoublepage

% ----------------------------------------------------------
% ELEMENTOS TEXTUAIS (Capítulos)
% ----------------------------------------------------------
\textual
% Elementos textuais com numeração arábica
\pagenumbering{arabic}
% Reinicia a contagem do número de páginas
\setcounter{page}{1}

% Inclui cada capitulo da Dissertação
\chapter{Introdução}\label{cap:introduction}

% Introdução
A sociedade está cada vez mais cercada e dependente de dispositivos baseados em computadores que movem nossas vidas em muitos aspectos. Estes dispositivos são projetados de modo que possam executar uma grande quantidade de softwares com diversas finalidades e o processo de criação destas aplicações, sendo ou não de forma completamente manual, está sujeita a erros.

% Contextualização
Ao tratarmos de desenvolvimento de software precisamos ter em mente que, apesar de computadores e softwares envolverem uma significativa parcela de automatização, consistência e velocidade em processos, a criação das ferramentas e recursos que moldam estes sistemas são ainda um processo humano que muito dependente deste fator \cite{human_factor_on_software_engineering}. Sem contar que atualmente existe uma enorme quantidade de ferramentas e recursos que podemos utilizar para criar softwares, cada um com suas características, pontos fracos e fortes. Além disso, esta pilha de tecnologias pode ser combinada de diversas maneiras de modo em que a integração entre elas também possam causar potenciais problemas.

Isso implica que ter clareza apenas das regras de negócio que uma aplicação deve atender não significa que o software desenvolvido para tal finalidade irá se comportar da maneira esperada. Também precisamos estar a par do poder das tecnologias que dispomos para podermos fazer melhor uso delas e explorar cada vez mais e melhor suas vantagens.

Uma falha ou defeito de um componente refere-se a uma condição anormal do mesmo, causada por problemas de lógica do sistema (no caso de software). Já o erro é a diferença entre o comportamento desejado/especificado e o real\cite{sandhof2006defeitos}. Ao longo deste trabalho, como analisamos os eventos gerados por defeitos ou erros em softwares, nos referimos aos dois da mesma forma.

Logo, caso surjam defeitos ou erros nessas aplicações, seja qualquer a razão, é importante que tenhamos em mente que significativas perdas podem ocorrer \cite{prejuizos_com_softwares} \cite{facebook_lost_millions_during_outage}. Seja em função da parada de uma parte fundamental ou importante da operação de uma empresa, ainda que se trate de um software intangível, mas também do tempo de manutenção e suporte, por exemplo. A implementação de novas funcionalidades precisa também ser considerada quando estamos planejando como será a arquitetura de uma solução, uma vez que software precisará estar em constante manutenção.

%Objetivo e Resultados
O principal objetivo deste trabalho, através dos resultados aqui demonstrados, é expressar a importância da adoção de boas práticas de programação, além de como a escolha da linguagem de programação empregada numa aplicação real pode ser crucial para mitigar os riscos da ocorrência de defeitos em tempo de execução em softwares em ambiente produtivo.

Primeiramente, é importante que tenhamos uma forma de classificar erros de engenharia de software no geral, para podermos entender quais de seus tipos são mais comumente encontrados. Para isso, buscamos na literatura científica quais referências podemos utilizar para termos uma classificação eficiente, que compreenda grande parte de suas possibilidades.

A classificação encontrada na etapa anterior foi aplicada em ocorrências capturadas via softwares de monitoramento de erros e exceções em uma aplicação de médio/grande porte, em termos de usuários ativos. Os erros analisados partem de exceções em Kotlin ou Java capturadas de forma proposital, ou não, para registrar um evento de erro capturado em tempo de execução. Isso facilita o processo de análise se comparado com a análise de \textit{issues} em repositórios públicos ou privados ou da correção aplicada a elas, por exemplo. Pois, observando as exceções lançadas, podemos ter uma ideia clara de qual é o defeito do ponto de vista de desenvolvimento do software em questão.

Analisando os defeitos, traçamos um paralelo entre as ocorrências observadas e o tipo de má prática que podem proporcionar maior risco de ocorrência destes erros. Baseado nelas, propomos outras abordagens, de baixo custo, que mitigam estes riscos. Serão propostas também técnicas que podem atuar de forma ativa na análise do código fonte da aplicação de modo acoplado a esteira de publicação. Desta forma, inúmeros problemas podem ser evitados em tempo de compilação e publicação, antes mesmo da revisão do código por pares durante o andamento da esteira de integração.

Por fim, comparamos duas linguagens muito parecidas (até interoperáveis entre si), que são Java e Kotlin, para entendermos quais recursos e paradigmas temos ao nosso dispor para evitarmos os defeitos analisados anteriormente. Além disso, utilizando o Kotlin como exemplo, aplicamos uma ferramenta de análise de estática de código \textit{linter} numa aplicação real, para entendermos quais benefícios temos com este tipo de processo e como podemos fazer para implementá-lo.

%Extrutura do Texto
Este trabalho é dividido em 6 partes. No \Cref{cap:cientific_library_research}, utilizando os procedimentos para uma revisão sistemática da literatura, estruturamos o levantamento de trabalhos que tratam a classificação de defeitos de software. No \Cref{cap:classification_on_real_world_app}, analisamos os dados das exceções capturadas por uma ferramenta de monitoramento em uma aplicação real. Utilizando a taxonomia obtida no \Cref{cap:cientific_library_research}, classificamos os dados e podemos ver quais são os erros mais comuns capturados nos dados. Já no \Cref{cap:most_common_errors_analysis}, propomos outras abordagens em relação às práticas que podem levar aos erros que foram encontrados nos dados analisados. Além disso, no \Cref{cap:linters}, nos aprofundamos na definição de ferramentas de análise estáticas, chamadas comumente de \textit{linters}, o que elas oferecem e um passo a passo de sua aplicação numa aplicação real, convertida de Java para Kotlin e os resultados atingidos com a análise da ferramenta. Por fim, no \Cref{cap:final_considerations} temos as considerações finais, explorando algumas conclusões dos dados obtidos e expondo os próximos trabalhos complementares a este.

% \include{capitulos/pt-br/02_metodologia}
\chapter{Revisão estruturada da literatura científica}\label{cap:cientific_library_research}

Para que pudéssemos iniciar nossa análise, precisávamos encontrar uma classificação consolidada de defeitos de software. Com isso, tivemos uma visão concreta da definição dos erros que buscamos evitar relacionados com a etapa de seu desenvolvimento e execução. Portanto, iniciamos buscando na literatura científica referências de autores que possam ter criado algum tipo de classificação de defeitos que possa ajudar neste processo.

Esta etapa consistiu nos seguintes passos:

\begin{itemize}
    \item Criação de uma \textit{string} de busca que será utilizada em ao menos um acervo digital de artigos científicos;
    \item Refinamento de uma classificação existente dos resultados por relevância;
    \item Seleção dos resultados que serão classificados através do título ou uma leitura rápida do resumo;
    \item Rotulação dos resultados pela classificação alcançada através da leitura do resumo;
    \item Leitura rápida dos artigos mais relevantes para reclassificação;
    \item Obter conclusões através dos artigos mais relevantes para embasar este estudo;
\end{itemize}

\section{\textit{String} de busca}

Para obter os resultados iniciais, construímos uma \textit{string} de busca considerando os principais termos deste estudo. Esta precisa conter a área e subáreas de pesquisa conforme a relevância. Sendo a área de pesquisa engenharia de software, no geral. O que buscamos são dados sobre defeitos que ocorram não apenas na fase de desenvolvimento, mas durante a execução da aplicação pelo usuário final. Em outras palavras, buscamos o que são comumente chamados de \textit{bugs}. Então, nossa área será ``software``, nossa primeira sub-área será ``engenharia`` ou ``desenvolvimento``, em referência a desenvolvimento ou engenharia de software. Depois teremos a sub-área dos defeitos, então ``\textit{bug}`` ou ``\textit{bugs}`` serão nossos próximos termos de pesquisa. 

Por fim, a última camada da pesquisa se trata da classificação dos defeitos. Então, teremos ``classificação`` como nossa última sub-área, uma vez que esta tem como fim encontrar os defeitos mais recorrentes. Logo, na \Cref{table:search_terms}, têm-se os termos relacionados.

\begin{table}[H]
    \centering
    \begin{threeparttable}
        \begin{tabular}{ c|c|c|c }
            \multirow{2}{*}{ Área de pesquisa } & \multicolumn{3}{ c }{ Subáreas } \\
            & 1 & 2 & 3 \\
            \hline
            software & development & bug* & classification \\
            & engineering & & \\
        \end{tabular}
        \caption{Divisão das áreas de pesquisa}
        \label{table:search_terms}
    \end{threeparttable}
\end{table}

As colunas da tabela estão conectadas numa lógica E e as linhas estão conectadas formando uma lógica OU. Então, teremos a seguinte \textit{string} para busca situada na \Cref{fig:search_string}.

\begin{figure}[H]
    \centering
    \begin{lstlisting}[numbers = none]
    TITLE-ABS-KEY ( software ) AND ( TITLE-ABS-KEY ( development ) OR TITLE-ABS-KEY ( engineering ) ) AND TITLE-ABS-KEY ( bug* ) AND TITLE-ABS-KEY ( classification )
    \end{lstlisting}
    \caption{\textit{String} de busca nos acervos científicos}
    \label{fig:search_string}
\end{figure}

Com a \textit{string} de busca pronta, a utilizamos para busca dos artigos que atendam nossos critérios.

\section{Classificação dos resultados obtidos}

Antes da busca ser realizada, será utilizada a metodologia do estudo de \cite{automated_tests_javascript} como base para classificação inicial dos resultados. Os critérios serão criados levando em consideração o objetivo deste estudo. Logo, teremos inicialmente o seguinte método de classificação por relevância demonstrado na \Cref{table:initial_relevance_and_criteria}

\begin{table}[H]
    \centering
    \begin{tabularx}{\textwidth}{ c l|l }
        \multicolumn{2}{ l| }{Relevância} & Critério \\
        \hline
        0 & Não relevante & Não deveria ser considerado na pré-análise \\
        1 & Pouco relevante & Faz algum tipo de análise da causa raiz de defeitos \\
        2 & Relevante & Faz alguma classificação entre tipos de defeitos \\
        3 & Muito relevante & Classifica e ordena os defeitos encontrados com maior frequência \\
    \end{tabularx}
    \caption{Relevâncias utilizadas para classificação dos artigos}
    \label{table:initial_relevance_and_criteria}
\end{table}

O próximo passo consiste na execução da busca para uma seleção binária, isto é, definir se determinado artigo será ou não analisado com maior profundidade, baseada nos títulos e uma rápida análise do resumo.

\section{Seleção dos resultados obtidos para classificação}

Após a execução da busca, utilizando a \textit{string} de busca criada anteriormente, na biblioteca Scopus \cite{scopus}, no dia 27 de março de 2023, foram obtidos 368 resultados.

Dos artigos encontrados, analisando o título de cada um deles, selecionamos 57 artigos que tinham potencial relevância para estudo para serem classificados através da metodologia criada no passo anterior. Com isso, já temos 311 artigos que serão rotulados com relevância 0 (não relevante).

\section{Classificação através da leitura dos resumos}

Analisando de forma rápida o resumo destes 57 artigos, 18 artigos se mostraram pouco relevantes. Enquanto 39 tiveram alguma relevância para o estudo, pois, além do estudo da causa raiz dos defeitos, algum tipo de classificação parecia ser feita, seja por severidade, tipo, \textit{Hidden Impact Bugs} (HIBs) ou defeitos comuns, reportes duplicados, entre outros.

É possível notar um grande volume de artigos que fazem algum tipo de classificação de defeitos. Porém, como a maioria das linhas de classificação não tem grande contribuição para este estudo, refinamos a classificação os artigos que se mostram relevantes.

Portanto, teremos a seguinte classificação representada na \Cref{table:refined_relevance_and_criteria}

\begin{table}[H]
    \centering
    \begin{tabularx}{\textwidth}{ c l|X }
        \multicolumn{2}{ l| }{Relevância} & Critério \\
        \hline
        0 & Não relevante & Não deveria ser considerado na pré-análise \\
        1 & Pouco relevante & Faz algum tipo de análise da causa raiz ou predição de defeitos \\
        2 & Média relevância & Classifica defeitos de qualquer forma que não seja do ponto de vista de desenvolvimento de software \\
        3 & Relevante & Faz alguma classificação entre tipos de defeitos do ponto de vista de desenvolvimento \\
        4 & Muito relevante & Classifica e ordena os defeitos encontrados do ponto de vista de desenvolvimento ranqueando-os pela frequência \\
    \end{tabularx}
    \caption{Relevâncias utilizadas para classificação dos artigos}
    \label{table:refined_relevance_and_criteria}
\end{table}

Analisando os resumos de cada artigos novamente, considerando este novo formato de classificação, apenas com os artigos considerados relevantes, tivemos 15, 7 e 17 artigos classificados com pouca, média e alta relevância, respectivamente.

Ao iniciar a leitura dos 17 artigos relevantes, 4 não puderam ser lidos, pois não foi obtido acesso aos mesmos, 11 não tinham relevância para este estudo e 2 artigos se mostraram relevantes para este contexto. Durante a leitura de um deles, um artigo citado também se mostra relevante.

Portanto, veja um resumo dos resultados na \Cref{table:reading_results}.

\begin{table}[H]
    \centering
    \begin{tabular}{ l|c }
        Descrição & Total de artigos \\
        \hline
        Artigos encontrados & 368 \\
        \hline
        Serão analisados pelo título & 57 \\
        \hline
        Pouco relevantes & 18 \\
        Relevantes & 39 \\
        \hline
        \multicolumn{2}{c}{\textit{Após reclassificação dos relevantes}}\\
        \hline
        Pouco relevantes & 15 \\
        Médio relevantes & 7 \\
        Relevantes & 17 \\
        \hline
        \multicolumn{2}{c}{\textit{Após leitura rápida dos relevantes}}\\
        \hline
        Não puderam ser acessados & -4 \\
        Sem relevância significativa & 11 \\
        Se mostraram relevantes & 2 \\
        \hline
        Adicionado via \textit{snowballing} & +1 \\
    \end{tabular}
    \caption{Classificações dos artigos e suas quantidades}
    \label{table:reading_results}
\end{table}

Resta agora analisar as conclusões que foram obtidas a partir de cada um destes três artigos.

\section{Análise dos artigos relevantes}

Apesar dos artigos selecionados não serem diretamente focados na classificação de defeitos ou busca de tipos de defeitos mais comuns, todos eles enfatizam a utilização de algoritmos de \textit{Machine Learning} para realizar a classificação automática de defeitos de software. Porém, a relevância destes para este estudo está relacionada a classificação utilizada, onde neles todos foi a classificação ortogonal de defeitos (ODC - \textit{Orthogonal Defect Classification}).

As conclusões a seguir terão foco neste segmento destes artigos. Porém, apenas no primeiro, analisamos a fundo a definição e conceitos que envolvem a ODC. Deste modo, concentramos nossos esforços no tema que mais fundamenta os resultados obtidos neste trabalho.

Como introduzido por \cite{automatic_odc_using_ml}, a classificação ortogonal de defeitos (ODC) é um \textit{framework} muito popular para classificação de defeitos de software. Ela considera diversos atributos de cada defeito para classificá-los. Outros estudos, que buscaram classificar defeitos de forma automatizada, levaram em consideração apenas um ou dois destes atributos, como tipo e impacto, por exemplo. Apesar de alguns deles estarem muito relacionados às ações tomadas pelos desenvolvedores para corrigir determinados defeitos, como os atributos qualificador e tipo de defeito, o modelo de \textit{machine learning} proposto no artigo considera sua maioria, como atividade, gatilho, impacto, alvo, tipo de defeito e qualificador. Por este motivo, esta classificação será utilizada ao longo deste trabalho.

A seção II do artigo fala sobre um \textit{background} do tema e trabalhos relacionados. Nela, a classificação ODC é definida como um processo analítico usado em desenvolvimento de software e análise do processo de testes para caracterizar defeitos de software. Ela permite extrair informações valiosas sobre os defeitos, além de fornecer \textit{insights} e diagnósticos de remediação no processo de engenharia de software.

A classificação é composta por oito atributos agrupados em duas seções: relatório aberto e fechado. Aqueles de relatório aberto se referem aos que se baseiam em informações disponíveis no momento de descoberta de um defeito, que são:

\begin{itemize}
    \item Atividade: se refere a atividade sendo executada no momento que o defeito surgiu, como, por exemplo, testes unitários;
    \item Gatilho: indica o que causou a manifestação do defeito, por exemplo, teste bloqueado;
    \item Impacto: o impacto causado a um usuário quando o defeito surgiu
\end{itemize}

Já os de relatório fechado, estão relacionados com a correção de determinados defeitos, pois dependem destas informações. Estes são:

\begin{itemize}
    \item Alvo: o objeto que foi o alvo da correção, por exemplo, uma classe ou conjunto de \textit{scripts} de \textit{build};
    \item Tipo de defeito: o tipo de alteração realizada para corrigir o problema, como a alteração da cláusula de um \textit{if}, por exemplo;
    \item Qualificador: descreve a característica do código anterior ao defeito antes da correção, por exemplo, se o código estava faltando, incorreto ou não deveria estar presente;
    \item Idade: o intervalo de tempo desde o momento em que o defeito surgiu;
    \item Origem: se refere a origem do defeito, se era um código pertencente uma dependência externa, ou um código do próprio projeto;
\end{itemize}

Como ponto central do nosso estudo é classificação dos erros já ocorridos em softwares, o principal atributo aqui observado é o tipo de defeito. Segundo a classificação ODC \cite{ibm_odc}, têm-se sete tipos de defeitos agrupados em duas categorias, que são:

\begin{itemize}
    \item Defeitos de fluxo e controle de dados
    \begin{itemize}
        \item Atribuição ou inicialização (A/I): valores atribuídos de forma incorreta ou não atribuídos;
        \item Verificação (C): validação incorreta ou faltante em condicionais que afetam o funcionamento do software;
        \item Algoritmo ou método (A/M): eficiência ou corretude de algoritmos que podem impactar o funcionamento, necessitando de uma reimplementação ou uma nova implementação distinta;
        \item Temporização ou serialização (T/S): a serialização de um objeto compartilhado foi feita de forma errônea ou não foi feita corretamente;
    \end{itemize}
    \item Defeitos estruturais
    \begin{itemize}
        \item Função, classe ou objeto (F/C/O): um erro que requer uma mudança mais drástica no design das classes que envolvem o software, sejam elas interfaces ou classes abstratas;
        \item Interface ou mensagens O-O (I/OOM): problemas de comunicação entre módulos, processos, objetos, sistemas, etc.;
        \item Relacionamento (R): problemas relacionados a associações entre procedimentos, estrutura de dados ou objetos.
    \end{itemize}
\end{itemize}

Enquanto isso, outros autores propõem classificações diferentes, como feito por \cite{ast_based_aproach_to_classifying_defects}, onde os erros são categorizados como erros de dados, computacionais, de interface e de controle/lógica. Porém, estas classificações não atendem tão bem aos objetivos deste trabalho, uma vez que estas não partem do ponto de vista de desenvolvimento de software.

A seção seguinte do artigo detalha os passos realizados durante o estudo. O mais importante para este trabalho é o passo inicial, de criação manual e pré-processamento do conjunto de dados. Isto porque tem certa semelhança com nossa fase de análise e coleta de dados e seus dados podem ser comparados com os classificados em nossa plataforma de monitoramento privada no \Cref{cap:classification_on_real_world_app}.

Os autores utilizaram três projetos \textit{open-source} de bancos de dados NoSQL, MongoDB, Cassandra e HBase. Destes projetos, foram extraídas 4096 \textit{issues} aleatoriamente, e foram manualmente classificadas por três pesquisadores. Veja na \Cref{table:occurrences_sum_classified_in_article} os resultados somados da verificação na tabela abaixo, onde as linhas representam os resultados originalmente obtidos pelo respectivo pesquisador, e as colunas são os tipos de defeitos obtidos durante a verificação.

\begin{table}[H]
    \centering
    \begin{tabularx}{\textwidth}{ X|c|c|c|c|c|c|c }
        \textbf{Tipo de defeito} & \textbf{A/M} & \textbf{F/C/O} & \textbf{C} & \textbf{I/OOM} & \textbf{A/I} & \textbf{T/S} & \textbf{R} \\
        \hline
        Algoritmo ou método (A/M) & 829 & 18 & 6 & 4 & 9 & 1 & \\
        \hline
        Função, classe ou objeto (F/C/O) & 17 & 221 &  & 1 &  &  & 1 \\
        \hline
        Verificação (C) & 17 &  & 146 &  &  & & \\
        \hline
        Interface ou mensagens O-O (I/OOM) & 6 & 4 & 2 & 116 & 1 &  & 1 \\
        \hline
        Atribuição ou inicialização (A/I) & 14 &  &  & 9 & 68 & & \\
        \hline
        Temporização ou Serialização (T/S) & 2 &  &  &  &  & 12 & \\
        \hline
        Relacionamento (R) & & & & & & & 2 \\
    \end{tabularx}
    \caption{Tabela com a soma das ocorrências classificadas no artigo}
    \label{table:occurrences_sum_classified_in_article}
\end{table}

Na \Cref{table:occurrences_sum_classified_in_article} dada, a diagonal principal representa os verdadeiros positivos. Destes tipos, de um total de 1394, o que se destacam são A/M (829 - 59\%), F/C/O (221 - 16\%), C (146 - 10\%) e I/OOM (81\%). A importância da análise destes dados para este trabalho é que as comparamos com as ocorrências de defeitos em outros projetos, para entendermos se há um padrão na frequência e volume de cada um de seus tipos.

O artigo de \cite{automatic_defect_categorization} também propõe um modelo de classificação de defeitos através da ODC e por isso tem sua relevância para este estudo. Apesar do conjunto de dados classificados manualmente que foi utilizado para treinamento do modelo ser relativamente pequeno, contando com 500 \textit{issues} apenas. Destas, 286 (57.2\%) são da família de defeitos de controle e fluxo de dados. Esta quantidade é consistente com o que foi encontrado nos outros conjuntos de dados neste e em outros estudos.

\chapter{Classificando erros mais comuns numa aplicação real}\label{cap:classification_on_real_world_app}

Como precisamos de uma classificação de erros mais próxima do ponto de vista de desenvolvimento de software, analisamos erros coletados através de uma ferramenta de monitoramento em um projeto privado de média/grande escala, a fim de analisar quais são as exceções mais frequentes num período de trinta (30) dias. Deste modo, podemos ter uma ideia melhor se estes erros poderiam ser evitados de acordo com a adoção de diferentes abordagens de desenvolvimento.

As ocorrências foram coletadas em uma aplicação Android, compilada utilizando o SDK do Android 8.1 (API 28), dando suporte a versão 5.0 (API 21) do Android como versão mínima. Foi desenvolvida em Java (JDK 8) e Kotlin (versão 1.5.10). O app é executado majoritariamente em um modelo de dispositivo próprio de uma companhia privada rodando o Android 7.1.1 (API 25) e alguns modelos executando o Android 6.0 (API 23). Porém, sua versão mínima é a 5.0 por motivos de retrocompatibilidade com outros dispositivos legados. O app importa diversas bibliotecas como dependências, sendo algumas destas públicas e outras privadas.

Nesta aplicação, temos implementada uma ferramenta BaaS (Backend as a Service) \cite{okta_backend_as_a_service} de monitoramento chamada Crashlytics \cite{crashlytics_home}, oferecida como um produto dentro do serviço Firebase, do Google. Esta é capaz de registrar ocorrências de exceções que foram ou não capturadas durante a execução do app, além de registros de logs ``não fatais``, isto é, que não comprometem a execução do app. Ainda que nem todas as falhas aqui capturadas sejam fatais, levamos em consideração que todas elas podem ter um impacto na visão do cliente, seja na experiência de usabilidade, quanto no que é esperado de alguma funcionalidade, ainda que alguns dos erros não interrompam o fluxo da aplicação. A \Cref{table:exceptions_classification} mostra os tipos e quantidades de ocorrências das exceções lançadas.

\begin{table}[H]
    \centering
    \begin{tabularx}{\textwidth}{ l|X|c|c }
        \textbf{Tipo de erro} & \textbf{Ocorrências em classes} & \textbf{Ocorrências} & \textbf{Usuários} \\
        \hline
        Erro de comunicação com o servidor & 8 & 1.120.357 & 137,126 \\
        Variável não inicializada & 25 & 16,104 & 6,914 \\
        Acesso inválido a vetor & 4 & 15,898 & 5,363 \\
        Erro de parseamento & 2 & 11,700 & 2,347 \\
        Erro de fluxo & 8 & 1009 & 589 \\
        Erro de tipo recebido & 1 & 499 & 434 \\
        Erro em aplicação externa & 2 & 363 & 282 \\
        Falha na leitura de arquivo & 1 & 135 & 135 \\
        Erro na inicialização & 1 & 122 & 107 \\
        Estouro de memória & 3 & 117 & 110 \\
        Acesso inválido a variável & 3 & 86 & 73 \\
        Erro de permissão & 1 & 35 & 4 \\
        Erro de banco de dados & 1 & 25 & 16 \\
    \end{tabularx}
    \caption{Exceções lançadas classificadas por tipo e suas quantidades de ocorrências}
    \label{table:exceptions_classification}
\end{table}

Podemos observar pela tabela \Cref{table:exceptions_classification}, que se encontra ordenada de forma decrescente pela quantidade de ocorrências em classes, que a maior quantidade de erros se deu pela falha de comunicação com o servidor. Isto é compreensível, por se tratar de uma aplicação móvel, onde muita oscilações de rede ocorrem. Porém, nas ocorrências seguintes, podemos observar um relevante número de falhas relacionadas a falhas de programação, sendo o caso dos defeitos de ``variável não inicializada`` e ``acesso inválido a vetor``, por exemplo. Os dois primeiros exemplos tiveram grande relevância entre os erros capturados e inclusive podem ser evitados através da adoção de diferentes práticas durante a fase de desenvolvimento.

Apesar de termos erros de todos os tipos de relevância em termos de quantidade de ocorrências, erros como de \textit{parseamento}, aplicação externa ou de conexão com o banco de dados, podem ser evitados mediante outras abordagens de tratamento de erros e coordenação do fluxo da aplicação, por exemplo. Portanto, a escolha da pilha de tecnologias adotada durante o planejamento, desenvolvimento e sustentação da aplicação demonstra uma relação direta com as ocorrências de defeitos nela observada.

Tendo em mente que a classificação dos defeitos do nosso repositório privado foi realizada apenas conforme o tipo de exceção lançada pela aplicação e não baseado na correção realizada para sanar os defeitos, podemos comparar o ranking dos erros mais comuns apenas como referência. Vejamos as ocorrências encontradas classificadas pelo ODC (veja o \Cref{cap:cientific_library_research}), na \cref{table:our_results_classified_by_odc}.

\begin{table}[H]
    \centering
    \begin{tabularx}{\linewidth}{ X|c|X|c|c|c }
        \textbf{Tipo de defeito} & \textbf{ODC} & \textbf{Ocorrências em classes} & \textbf{\%} & \textbf{Ocorrências} & \textbf{\%} \\
        \hline
        Variável não inicializada & \multirow{3}{*}{ A/I } & 25 & \multirow{3}{*}{ 55,77\% } & 16,104 & 
        \multirow{3}{*}{ 35,39\% } \\
        Acesso inválido a variável & & 3 & & 86 & \\
        Erro na inicialização & & 1 & & 122 & \\
        \hline
        Erro de fluxo & A/M & 8 & 15,38\% & 1009 & 2,19\% \\
        \hline
        Erro em aplicação externa & \multirow{2}{*}{I/OOM} & 2 & \multirow{2}{*}{5,77\%} & 363 & \multirow{2}{*}{0,84\%} \\
        Erro de banco de dados & & 1 & & 25 \\
        \hline
        Erro de parseamento & T/S & 2 & 3,85\% & 11,700 & 25,38\% \\
        \hline
        Erro de tipo recebido & F/C/O & 1 & 1,92\% & 499 & 1,08\% \\
        \hline
        Falha na leitura de arquivo & R & 1 & 1,92\% & 135 & 0,29\% \\
        \hline
        Acesso inválido a vetor & \multirow{2}{*}{C} & 4 & \multirow{2}{*}{9,62\%} & 15,898 & \multirow{2}{*}{34,57\%} \\
        Erro de permissão & & 1 & & 35 & \\
        \hline
        Estouro de memória & - & 3 & 5,77\% & 117 & 0,26\% \\
        \hline
        \multicolumn{2}{l|}{\textbf{Total}} & \textbf{52} & \textbf{100\%} & \textbf{46093} & \textbf{100\%} \\
    \end{tabularx}
    \caption{Parcela de cada tipo de defeito nas ocorrências analisadas}
    \label{table:our_results_classified_by_odc}
\end{table}

Pode-se notar na \Cref{table:our_results_classified_by_odc} que os defeitos mais frequentes foram de atribuição ou inicialização (A/I), seguido do algoritmo ou método, tipo A/M (algoritmo ou método), e, por fim, temos os erros de interface ou mensageria (tipo I/OOM).

De qualquer forma, erros de tipos como de algoritmo ou método, função, classe ou objeto, verificação, atribuição ou inicialização, por exemplo, podem ser mitigados com a adoção de diferentes abordagens, paradigmas de programação ou ferramentas de verificação (conforme detalharemos nas seções seguintes deste estudo). Tanto nos conjuntos de dados da pesquisa preliminar, quando nos dados analisados neste capítulo, estes tipos de defeito somam mais de 80\% das ocorrências. Isto reforça a hipótese deste trabalho de que adotar diferentes práticas de programação podem ser úteis para mitigar o risco de defeitos durante o processo de desenvolvimento.

\chapter{Análise dos erros mais frequentes numa aplicação real}\label{cap:most_common_errors_analysis}

Baseado nos erros mais comuns levantados no \Cref{cap:classification_on_real_world_app}, vejamos quais erros podem ser sanados primeiramente através da utilização de diferentes linguagens ou paradigmas de programação. Tenhamos em mente que os erros classificados foram coletados através do monitoramento de uma aplicação desenvolvida para Android, utilizando Kotlin na maioria do seu código-fonte e Java em algumas classes legadas. Com isso, analisamos se utilizando uma dessas linguagens existem meios de evitar ou conter estes erros que podem ser comumente evitados por desenvolvedores.

É importante destacar também que esta aplicação é executada em dispositivos móveis. Logo, alguns dos erros não partem de erros na fase de desenvolvimento, mas sim de fatores externos, como ``Erro de comunicação com o servidor``, por exemplo. Estes tipos de defeitos não serão analisados nesta etapa do projeto.

\section{Variável não inicializada}

\subsection{Java}

O erro de variável não inicializada consiste no acesso do conteúdo de uma variável antes que ela tenha sido devidamente inicializada, isto é, antes que um conteúdo válido seja atribuído a ela. Observe um exemplo abaixo utilizando Java para simular o erro na \Cref{fig:java_npe_code}.

\begin{figure}[H]
    \centering
    \begin{lstlisting}[language=Java]
class NullPointerExceptionSimulation {
    public static void main(String[] args) {
        Result test = null;

        System.out.print(test.data().toString())
    }
}
    \end{lstlisting}
    \caption{Exemplo de código que lança uma exceção do tipo \textit{NullPointerException}}
    \label{fig:java_npe_code}
\end{figure}

Note que durante a execução deste código uma exceção do tipo \textit{NullPointerException} seria lançada na linha 5. Pois durante a chamada da função \textit{data()} do objeto de tipo \textit{Result}, nenhum conteúdo foi atribuído a variável que o contém.

Algumas linguagens possuem mecanismos para impedir este tipo de erro em tempo de compilação, para evitar que o desenvolvedor possa executar a aplicação com o risco de uma variável referenciada sem que seja atribuído um valor a mesma.

\subsection{Kotlin}\label{subsec:most_common_error_uninit_variable_kotlin}

Kotlin é uma linguagem de programação de código aberto \cite{kotlin_source} que possui interoperabilidade com a linguagem Java \cite{kotlin_get_started}, porém, conta com diversas melhorias e novas funcionalidades. Nela, por padrão, todos os tipos não são anuláveis, ou seja, não podem receber um valor nulo \cite{kotlin_nullable_types}. A linguagem tem o objetivo de eliminar o NPE (\textit{NullPointerException}) \cite{kotlin_null_safety} por completo. Além disto, seu compilador possui suporte a utilização de variáveis imutáveis, isto é, propriedades apenas com acesso de leitura que devem ter seu valor atribuído em sua inicialização \cite{kotlin_variables}.

Note que, no exemplo da \Cref{fig:kotlin_not_init_variable}, são utilizados dois tipos de variáveis, variáveis do tipo \textit{val} e \textit{var}. O primeiro possui a característica de imutabilidade, citado anteriormente. Já tipo \textit{var} pode ter seu conteúdo alterado a qualquer momento. Utilizando qualquer um dos dois tipos, o programador não consegue, sem sobrepor este comportamento com abordagens que veremos a seguir, não inicializar uma variável no momento da criação e nem mesmo atribuir a ela um valor nulo direta e explicitamente. Dito isso, o código na \Cref{fig:kotlin_not_init_variable} não compila por conta de erros nas linhas 2 e 3.

\begin{figure}[H]
    \centering
    \begin{lstlisting}[language=kotlin]
class UninitializedVariableError {
    private val nonChangeableValue: Int
    private var changeableVariable: Int = null

    fun execute(){
        print("Non changeable value is $nonChangeableValue")
        print("Changeable value is $changeableVariable")
    }
}
    \end{lstlisting}
    \caption{Exemplo de código incorreto de variável não inicializada}
    \label{fig:kotlin_not_init_variable}
\end{figure}

Já o código na \Cref{fig:kotlin_init_variable} compila normalmente e pode ser executado sem problemas.

\begin{figure}[H]
    \centering
    \begin{lstlisting}[language=kotlin]
class UninitializedVariableError {
    private val nonChangeableValue: Int = 0
    private var changeableVariable: Int = 1

    fun execute(){
        print("Non changeable value is $nonChangeableValue")
        print("Changeable value is $changeableVariable")
    }
}
    \end{lstlisting}
    \caption{Exemplo de código funcional após inicializar variáveis}
    \label{fig:kotlin_init_variable}
\end{figure}

Entretanto, este comportamento do compilador pode ser sobreposto utilizando o operador \textit{?}. Quando utilizado na definição explícita de tipos, indica que determinado tipo também será anulável. Porém, ao acessarmos o valor de variáveis anuláveis, o compilador requererá uma entre duas das verificações.

\textbf{chamada segura (\textit{safe call}):} utilizando o operador \textit{?} após a leitura da variável, se caso o valor da variável for nulo a chamada não será feita. Veremos exemplos deste caso na \Cref{fig:kotlin_null_variable_without_safe_call}, que não é aceito pelo compilador.

\begin{figure}[H]
    \centering
    \begin{lstlisting}[language=Kotlin]
class UninitializedVariableError {
    private val nullableValue: Int? = null
    private var nonNullableVariable: Int = 0

    fun execute(){
        print("Non nullable value is ${nullableValue.toFloat()}")
        print("Nullable value is $nonNullableVariable")
    }
}
    \end{lstlisting}
    \caption{Exemplo de chamada de uma variável nula sem \textit{safe call}}
    \label{fig:kotlin_null_variable_without_safe_call}
\end{figure}

Já na \Cref{fig:kotlin_null_variable_with_safe_call}, temos um exemplo aceito pelo compilador.

\begin{figure}[H]
    \centering
    \begin{lstlisting}[language=Kotlin]
class UninitializedVariableError {
    private val nullableValue: Int? = null
    private var nonNullableVariable: Int = 0

    fun execute(){
        print("Non nullable value is ${nullableValue?.toFloat()}")
        print("Nullable value is $nonNullableVariable")
    }
}
    \end{lstlisting}
    \caption{Exemplo de chamada de uma variável nula com \textit{safe call}}
    \label{fig:kotlin_null_variable_with_safe_call}
\end{figure}

Note que se a variável \textit{nullableValue} estiver atribuído o valor nulo, \textit{toFloat} não será executado e o valor \textit{null} será passado para ser convertido para \textit{String?} (String anulável).

\textbf{assegurar que a variável não é nula (operador \textit{!!})}: este operador indica que o programador garante que a variável não terá valor nulo no momento da chamada e então, uma chamada segura não será necessária. Porém, se este contrato não for cumprido, uma exceção do tipo \textit{NullPointerException} (que chamaremos de NPE, por conveniência) será lançada. Vejamos alguns exemplos de sua utilização e de códigos que lançariam ou não uma NPE.

Note que o código na \Cref{fig:kotlin_not_init_variable_invalid_null_assert} dispara uma NPE quando executado na linha 6, uma vez que \textit{nullableValue} possui valor nulo quando \textit{toFloat} é chamada.

\begin{figure}[H]
    \centering
    \begin{lstlisting}[language=Kotlin]
class UninitializedVariableError {
    private val nullableValue: Int? = null
    private var nonNullableVariable: Int = 0

    fun execute(){
        print("Non nullable value is ${nullableValue!!.toFloat()}")
        print("Nullable value is $nonNullableVariable")
    }
}
    \end{lstlisting}
    \caption{Exemplo de chamada de uma variável nula com \textit{null assert}, que dispara exceção}
    \label{fig:kotlin_not_init_variable_invalid_null_assert}
\end{figure}

Já no código da figura \ref{fig:kotlin_not_init_variable_valid_null_assert}, nenhuma exceção é lançada, já que a variável \textit{nullableValue} possui um valor diferente de nulo.

\begin{figure}[H]
    \centering
    \begin{lstlisting}[language=Kotlin]
class UninitializedVariableError {
    private var nullableVariable: Int? = null
    private var nonNullableVariable: Int = 0

    fun execute(){
        nullableVariable = Random(1000).nextInt()
        
        print("Non nullable value is ${nullableVariable!!.toFloat()}")
        print("Nullable value is $nonNullableVariable")
    }
}
    \end{lstlisting}
    \caption{Exemplo de chamada válida de uma variável anulável com \textit{null assert}}
    \label{fig:kotlin_not_init_variable_valid_null_assert}
\end{figure}

Existe ainda uma outra forma de sobrepor a verificação de atribuição durante a declaração de variáveis não anuláveis, que são as variáveis de inicialização posterior (ou \textit{lateinit variables}, em inglês). Sua utilização é permitida apenas para variáveis mutáveis (definidas utilizando a palavra \textit{var}), onde pode-se adicionar a palavra reservada \textit{lateinit} antes e indicar que esta variável será inicializada posteriormente.

A leitura de uma variável do tipo \textit{lateinit var} que ainda não foi inicializada dispara uma exceção do tipo \textit{UninitializedPropertyAccessException}. Veremos dois exemplos onde o primeiro (\Cref{fig:kotlin_uninitialized_lateinit_var}) dispara a exceção anterior e, o segundo (\Cref{fig:kotlin_lateinit_variable_initialized}), não:

\begin{figure}[H]
    \centering
    \begin{lstlisting}[language=Kotlin]
class UninitializedVariableError {
    private lateinit var lateInitVariable: List<Int>

    fun execute(){
        print("Non nullable value is ${lateInitVariable.first()}")
    }
}
    \end{lstlisting}
    \caption{Exemplo de variável do tipo \textit{lateinit} é lida antes de ser inicializada}
    \label{fig:kotlin_uninitialized_lateinit_var}
\end{figure}

\begin{figure}[H]
    \centering
    \begin{lstlisting}[language=Kotlin]
class UninitializedVariableError {
    private lateinit var lateInitVariable: List<Int>

    fun execute(){
        lateInitVariable = listOf(0, 1, 2)
        
        print("Non nullable value is ${lateInitVariable.first()}")
    }
}
    \end{lstlisting}
    \caption{Exemplo de variável do tipo \textit{lateinit} utilizada de forma válida}
    \label{fig:kotlin_lateinit_variable_initialized}
\end{figure}

Repare que a implementação dos dois métodos que garantem uma segurança em relação a valores nulos, sem requerer muito esforço para termos uma boa prática de implementação.

\section{Acesso inválido a vetor}

\subsection{Java}

O erro de acesso inválido a um vetor é lançado quando, em um vetor de tamanho \textit{x}, tentamos acessar qualquer posição maior ou igual a \textit{x} (assumindo que as posições do vetor começam em zero). Vejamos um exemplo simples em Java na \Cref{fig:java_invalid_array_access}.

\begin{figure}[H]
    \centering
    \begin{lstlisting}[language=Java]
public class ArrayOutOfBoundsError {
    public static void execute(){
        int[] list = new int[] { 0, 1, 2 };

        System.console().printf("%d", list[4]);
    }
}
    \end{lstlisting}
    \caption{Exemplo de um acesso inválido ao vetor em Java}
    \label{fig:java_invalid_array_access}
\end{figure}

Vejamos quais recursos temos a disposição utilizando outras linguagens de programação sem a utilização de nenhum tipo de \textit{plugin} dessas linguagens para mitigarmos o risco do tipo de exceção \textit{ArrayOutOfBoundsException}.

\subsection{Kotlin}

Em Kotlin, por padrão, os vetores são invariantes. Isto é, todo o vetor precisa ter um tipo para que ele possa somente receber elementos daquele tipo específico \cite{kotlin_arrays}. Logo, ao criarmos um vetor de números inteiros, podemos adicionar somente elementos que sejam números inteiros ao vetor. 

Considerando o suporte nativo da linguagem, assim como Java, Kotlin não dispõe de nenhum mecanismo adicional que impeça o programador de tentar realizar um acesso inválido a um vetor. Existem validações realizadas pela IDE, mas ainda assim, um código ou lógica errôneos ainda podem ser compilados e executados. Veja um exemplo na \Cref{fig:kotlin_invalid_array_access}.

\begin{figure}[H]
    \centering
    \begin{lstlisting}[language=Kotlin]
class ArrayOutOfBoundsError {
    fun execute(){
        val array = arrayOf(0, 1, 2)

        System.console().printf(array[4].toString())
    }
}
    \end{lstlisting}
    \caption{Exemplo de um acesso inválido ao vetor em Kotlin}
    \label{fig:kotlin_invalid_array_access}
\end{figure}

 Porém, a linguagem oferece algumas funções extras nativas para facilitar a verificação da validade do índice acessado. Como vimos anteriormente, o operador \textit{[]} ainda é utilizado para acessar um índice específico do vetor. Este operador é análogo à função \textit{get} de elementos do tipo \textit{Array}. 
 
 Apesar disso, esta função possui duas variantes chamadas \textit{getOrNull} \cite{kotlin_get_or_null} e \textit{getOrElse} \cite{kotlin_get_or_null}. A primeira recebe um parâmetro, que é o índice desejado. Caso o índice seja inválido, o valor nulo é retornado, para que possa ser tratado de uma das formas relatadas no capítulo anterior.

 Já a função \textit{getOrElse} recebe dois parâmetros: o primeiro também é o índice que deve ser acessado, enquanto o segundo é uma função lambda que recebe um valor inteiro, que corresponde ao índice que tentou ser acessado. Ela retorna um valor do mesmo tipo do vetor, para que seja utilizado caso o índice acessado inicialmente seja inválido.

 Vejamos um exemplo simples da última função na \Cref{fig:kotlin_get_or_else_array_access}.

\begin{figure}[H]
    \centering
    \begin{lstlisting}[language=Kotlin]
class Fibonacci {
    fun preCalculated(i: Int): Int = 
        intArrayOf(1, 1, 2, 3, 5, 8, 13, 21).getOrElse(i) { calculate(it) }

    companion object {
        fun calculate(i: Int): Int = 
            if (i <= 2) 1 else calculate(i - 1) + calculate(i - 2)
    }
}
    \end{lstlisting}
    \caption{Exemplo de um utilização da função \textit{getOrElse} em Kotlin}
    \label{fig:kotlin_get_or_else_array_access}
\end{figure}

 A função anterior serve para retornamos valores da série de Fibonacci. Suponhamos que os 6 valores iniciais sejam os mais utilizados. Com isso, podemos ter estes valores pré-calculados, e, caso nossa lista de \textit{cache} não contenha o valor acessado, podemos utilizar o índice para calculá-lo em tempo de execução. Não temos ganhos significativos neste exemplo, mas, se tratássemos de um cálculo mais complexo, este tipo de abordagem poderia ser útil.

\section{Erros de fluxo de algoritmos}

Erros de algoritmo ou método são defeitos causados durante a aplicação errônea das regras de negócio na aplicação. Isto é, alguma falha no tratamento de algum cenário pode gerar uma falha no funcionamento do algoritmo que o impede de continuar ou dispara uma série de comportamentos inesperados na aplicação.

Entretanto, este tipo de defeito não fica limitado a este cenário. Fatores externos também podem ocasionar este tipo de falha, ainda que todas as regras de negócio estejam cobertas e tratadas da forma correta. Quando algum evento fora do esperado ocorre, a aplicação pode demonstrar um comportamento não esperado ou até mesmo não ser capaz de continuar executando.

Vejamos dois métodos de tratamento destes cenários a seguir.

\section{Exceções}

No primeiro caso, onde a instrução atual interrompe a execução da aplicação, temos o que é chamado de exceção. Uma exceção é um evento que ocorre durante a execução de um programa por conta da disrupção do fluxo normal ou esperado de suas instruções \cite{java_exception_definition}.

A linguagem Java possui três tipos de exceções, são elas:

\begin{itemize}
    \item \textit{Checked exceptions}: são exceções criadas para representar eventos excepcionais dos quais uma aplicação bem escrita ainda consegue se recuperar e seguir a execução. Por exemplo, durante a busca por um arquivo por um dado caminho (\textit{path}), há possibilidade do mesmo não ser encontrado. Neste caso, uma exceção do tipo \textit{FileNotFoundException} é lançada e qualquer aplicação bem escrita deve antecipar este cenário;
    \item Erros (\textit{errors}): erros normalmente representam eventos \textbf{externos} a aplicação das quais ela não pode antecipar ou se recuperar. Imagine que ao abrir um arquivo, o hardware sofre uma falha repentina e o arquivo não pode mais ser acessado durante a leitura. Então, um erro do tipo \textit{java.io.IOError} seria lançada;
    \item \textit{Runtime exceptions}: são exceções \textbf{internas} a aplicação que representam algum evento inesperado e irrecuperável. Normalmente, representam alguma falha na lógica ou fluxo da aplicação. Exceções do tipo \textit{NullPointerException} podem ser um exemplo deste tipo.
\end{itemize}

Nesta linguagem existe também um princípio chamado \textit{Catch or Specify Requirement} \cite{java_catch_or_specify_requirement}. Este é implementado no compilador e exige a aplicação de dois conceitos para a chamada de um método que pode explicitamente lançar uma exceção, que são:

\begin{itemize}
    \item um bloco \textit{try} em torno o método que pode lançar a exceção que capture e trate aquela exceção (\Cref{fig:java_catching_exception});
        \begin{figure}[H]
            \centering
            \begin{lstlisting}[language=Java]
public class CatchingExample {
    public static void catching() {
        try {
            throwsException();
        } catch(Exception e) {
            // Do something with the exception
        }
    }
}
            \end{lstlisting}
            \caption{Exemplo da captura de exceção Java lançada por um método}
            \label{fig:java_catching_exception}
        \end{figure}
    \item o método que cerca a possível exceção precisa declarar que aquele tipo de exceção pode ser lançado (\Cref{fig:java_declaring_exception});
        \begin{figure}[H]
            \centering
            \begin{lstlisting}[language=Java]
public class SpecifyingExample {
    public static void specified() throws IOException {
        if (true) throw new IOException();
    }
}
            \end{lstlisting}
            \caption{Exemplo de declaração de exceção em Java}
            \label{fig:java_declaring_exception}
        \end{figure}
\end{itemize}

Porém, apenas as exceções do tipo \textit{checked exceptions} são contempladas pelo princípio acima. Ainda assim, exceções podem ser um caminho útil para controlar o fluxo de aplicações. Porém, não é a única maneira de ter controles sobre eventos esperados ou não durante a execução da aplicação.

Apesar da utilização de exceções serem uma boa alternativa para controlar eventos não esperados numa aplicação, existe a possibilidade de optar por algoritmos que não disparam exceções de forma esperada.

\section{Utilização de classes do tipo \textit{Result}}

Utilizando orientação a objetos como base de exemplo, tome como exemplo a interface do tipo \textit{Result}, que representa um resultado de uma determinada operação, na \Cref{fig:java_result_interface}.

\begin{figure}[H]
    \centering
    \begin{lstlisting}[language=Java]
public interface Result {
    boolean isSuccess();
}
    \end{lstlisting}
    \caption{Exemplo de interface de resultados em Java}
    \label{fig:java_result_interface}
\end{figure}

Esta interface possui um único método que indica se o resultado é de sucesso ou de falha. Logo, implementaremos esta interface de duas formas que representem cada um dos casos (\Cref{fig:java_success_result_implementation}).

\begin{figure}[H]
    \centering
    \begin{lstlisting}[language=Java]
public class Success<T> implements Result {
    private final T data;

    public Success(T data) {
        this.data = data;
    }

    @Override
    public boolean isSuccess() {
        return true;
    }

    public T getData() {
        return data;
    }
}
    \end{lstlisting}
    \caption{Exemplo de implementação de uma interface de resultado que representa o resultado de sucesso}
    \label{fig:java_success_result_implementation}
\end{figure}

Note que \textit{Success} é uma classe que carrega um tipo genérico que indica o tipo de dados que carrega o resultado de sucesso, dentro da propriedade \textit{data}. E, sendo um resultado de sucesso, o método \textit{isSuccess} sempre returna o valor verdadeiro.

Já a implementação do tipo \textit{Fail}, na \Cref{fig:java_fail_result_implementation}, carrega uma propriedade \textit{error} que representa o erro capturado na execução de terminada tarefa.

\begin{figure}[H]
    \centering
    \begin{lstlisting}[language=Java]
public class Fail implements Result {
    private final Error error;
    
    public Fail(Error error) {
        this.error = error;
    }
    
    @Override
    public boolean isSuccess() {
        return false;
    }

    public Error getError() {
        return error;
    }
}
    \end{lstlisting}
    \caption{Exemplo de declaração de exceção em Java}
    \label{fig:java_fail_result_implementation}
\end{figure}

Este tipo de implementação é inspirada nos \textit{Monads} que são comumente encontrados em linguagens funcionais. \textit{Monads} são, basicamente, uma forma de estruturar uma sequência de computações e valores utilizando valores fortemente tipados \cite{haskell_monads}.

Quando utilizamos resultados que são fortemente tipados, o compilador pode se encarregar de garantir que um tratamento para todos os tipos possíveis sejam implementados nos algoritmos. Com isso, fica mais difícil de termos eventos inesperados em tempo de execução que não foram tratados durante a implementação do algoritmo.

\subsection{Explorando tipo Result utilizando pattern matching}

A linguagem Kotlin possui uma expressão chamada \textit{when} que define uma expressão condicional de múltiplas ramificações. Dentre estas ramificações, a versão atual de seu compilador é capaz de garantir que todas as suas possibilidades sejam cobertas. Para isso, esta expressão requer uma das ramificações tenha o valor \textit{else}, quando nenhuma outra possibilidade listada é atendida \cite{kotlin_when_expression}.

Porém, se a expressão em questão tem um valor passado a ela e sendo este valor de uma classe do tipo \textit{sealed} ou \textit{enum}, então o compilador é capaz de avaliar se todas as possibilidades foram cobertas em tempo de compilação do tratamento das ramificações da expressão \textit{when}.

Por exemplo, tome a classe da \Cref{fig:kotlin_result_implementation} que executa uma determinada operação.

\begin{figure}[H]
    \centering
    \begin{lstlisting}[language=Kotlin]
object SomeComplexOperation {
    fun execute(): Result {
        return try {
            /* A series of complex operations */
            Result.Success(true)
        } catch (e: Exception) {
            Result.Fail(e)
        }
    }
}
    \end{lstlisting}
    \caption{Exemplo de utilização da classe de resultado em Kotlin}
    \label{fig:kotlin_result_implementation}
\end{figure}

Tome também uma outra abordagem de classes de resultado, na \Cref{fig:kotlin_result_sealed_type}.

\begin{figure}[H]
    \centering
    \begin{lstlisting}[language=Kotlin]
sealed class Result {
    data class Success<T>(val data: T) : Result()
    data class Fail(val exception: Exception) : Result()
}
    \end{lstlisting}
    \caption{Exemplo de implementação de classe \textit{sealed} em Kotlin como resultado}
    \label{fig:kotlin_result_sealed_type}
\end{figure}

Note que o resultado da execução da operação acima pode ser um resultado de sucesso ou de erro. Utilizando uma expressão \textit{when} o compilador é capaz de garantir que os dois processos são cobertos, sem a necessidade da utilização de uma ramificação de tipo \textit{else}. Veja \Cref{fig:kotlin_result_class_example}.

\begin{figure}[H]
    \centering
    \begin{lstlisting}[language=Kotlin]
class Program {
    fun main(args: String[]) {
        val result = SomeComplexOperation.execute()

        when(result) {
            is Result.Success<*> -> {
                /* Do something with the data inside the success object */
            }
            is Result.Error -> {
                /* Do something with the error */
            }
        }
    }
}
    \end{lstlisting}
    \caption{Exemplo de utilização de classes de resultado em Kotlin}
    \label{fig:kotlin_result_class_example}
\end{figure}

Desta forma, podemos construir algoritmos que não lançam exceções, mas sim devolvem resultados de qualquer que seja o tipo. Observe que a implementação da classe \textit{Result} não está limitada a tipos genéricos de sucesso ou falha. Pode-se implementar inúmeros tipos de resultados e garantir que todos eles sejam tratados utilizando este tipo de abordagem.

É evidente que as abordagens descritas neste capítulo podem não sanar defeitos em softwares de forma direta. Porém, algumas delas previnem que eventos não esperados ou comportamentos errôneos não sejam tratados da forma correta pela aplicação. Algumas delas ainda impedem que um comportamento errático seja propagado adiante durante a execução.

Ainda que algumas delas adicionem uma certa complexidade ao desenvolvimento, o propósito final é impedir que outros desenvolvedores que contribuem para estas aplicações possam empregar menos esforços prevendo e tratando todos os casos de erro. Conforme estas práticas se tornam comuns, a própria estrutura da aplicação torna inevitável que resultados diferentes do esperado sejam previstos ou devidamente tratados.

Por estes motivos, o ganho obtido com algumas abordagens não pode ser esperado a curto prazo. Mas se este é o objetivo, existem outras ferramentas que podem realizar uma análise imediata da aplicação. E, dependendo de como utilizada, tem uma ação semelhante as estratégias abordadas aqui. Sendo ela de garantir que o código-fonte atual siga determinadas regras e que nenhum incremento não conforme seja feito a aplicação. 

\chapter{Ferramentas de detecção de erros}\label{cap:linters}

 Além das funcionalidades dispostas em linguagens e paradigmas de programação, existem outras ferramentas que podem nos ajudar a melhorar a qualidade do código, indicando possíveis erros e vulnerabilidades.
 
 Os \textit{linters} são ferramentas de análise estáticas que apoiam os desenvolvedores detectando possíveis erros de código, má práticas de programação, violação da convenção de determinadas linguagens e problemas de formatação do código ao mesmo tempo que oferece dicas práticas de melhorias nas implementações para mitigar a ocorrência de erros e inconsistências durante o desenvolvimento \cite{analysing_linter_usage_and_warning}.

Estas ferramentas estão integradas diretamente nas IDEs mais utilizadas, outras podem ser adicionadas manualmente via \textit{plugins} ou com algum gerenciador de \textit{build}, como no caso do \textit{detekt}, para Kotlin \cite{detekt_gettint_stated_gradle}.

Os \textit{linters} possuem conjuntos de regras da qual a análise irá apontar violações das mesmas. Estas regras estão relacionadas com os erros ou pontos de atenção mais comuns de determinada linguagem. O detekt, por exemplo, foi criado para aplicações em Kotlin e conta com alguns conjuntos de regras, como:

\begin{itemize}
    \item regras de comentários;
    \item regras de complexidade;
    \item regras de corrotinas (rotinas assíncronas utilizadas na linguagem Kotlin \cite{kotlin_coroutines});
    \item regras de blocos vazios;
    \item regras de exceções;
    \item regras de formatação;
    \item regras de bibliotecas (dependências);
    \item regras de nomenclatura;
    \item regras de desempenho;
    \item regras de potenciais defeitos (\textit{bugs});
    \item regras de criação de regras;
    \item regras de estilo de código;
\end{itemize}

Estes conjuntos de regras podem ser suprimidos ou customizados, através da criação de novas regras, inclusive \cite{detekt_config}. Estas ferramentas exportam relatórios com o resultado das análises realizadas seguindo a configuração realizada para determinado projeto para correção e melhoria contínua do software.

Além disso, os \textit{linters} se integram bem com o processo de desenvolvimento e implementação contínua (CI/CD), uma vez que sua execução pode ser implementada em esteiras de validação e publicação, a fim de validar que o código alterado atende os requisitos previamente definidos.

\textit{Linters} não estão relacionados apenas a erros e vulnerabilidades, mas também em relação à qualidade do código em si, uma vez que estes podem receber regras que visam apenas a qualidade visual do código a fim de tornar o código mais legível.

Sendo assim, cada projeto pode ter seu próprio conjunto de regras que ajudará a manter a qualidade do código, visando manter a qualidade, segurança, facilidade de manutenção e mitigação da possibilidade de erros em produção.

Dentro das regras que é possível definir utilizando o \textit{detekt}, por exemplo, podemos definir a partir de qual severidade de violação encontrada no código pode fazer com que a construção de um artefato falhe. Desta forma, podemos impedir que uma publicação automatizada falhe, por exemplo, em caso de violação de alguma das regras. Logo, através deste processo, podemos mitigar ainda mais o risco de ocorrências de falhas em produção.

Vejamos um exemplo prátivo da implementação deste tipo de ferramenta.

\section{Implementação de um linter em um projeto em Kotlin}\label{sec:linter_implementation}

Como uma prova de conceito, analisamos como pode ser feita a implementação de um \textit{linter} em um projeto existente em Kotlin. Garantimos também que a execução do processo esteja vinculado e seja dependência para que um artefato executável do projeto seja criado.

O projeto utilizado será uma implementação básica da ferramenta ZooKeeper\footnote{\url{https://github.com/vinicreis/zookeeper-kotlin}}, da Apache. Esta ferramenta disponibiliza um servidor de código aberto que tenha alta disponibilidade através de um serviço distribuído e sincronizado, que pode conter informações de configuração, nomenclatura, entre outras, distribuída entre várias instâncias que pode ser utilizadas em aplicações distribuídas \cite{apache_zookeeper_home}.

O projeto utilizado está implementado completamente em Kotlin \cite{kotlin_home} (1.9.20) e conta com o Gradle \cite{gradle_home} como ferramenta de \textit{build} (8.4). Inicialmente implementado em Java, o projeto foi convertido para Kotlin para tirar proveito do \textit{linter} na busca de melhorias e possíveis vulnerabilidades na implementação inicial.

A arquitetura conta com uma implementação de múltiplos módulos, sendo estes módulo de domínio de toda a aplicação \textit{model}\footnote{\url{https://github.com/vinicreis/zookeeper-kotlin/tree/main/model}}, um controlador (\textit{controller}\footnote{\url{https://github.com/vinicreis/zookeeper-kotlin/tree/main/client}}) - que seria o servidor central que coordena a sincronização entre os nós do serviço, o nó (\textit{node}\footnote{\url{https://github.com/vinicreis/zookeeper-kotlin/tree/main/client}}), que hospedam os dados para ampliar a disponibilidade dos dados e um cliente (\textit{client}) - que é capaz de realizar consultas e operações de salvamentos de dados no formato chave/valor.

Existe também um módulo \textit{buildSrc}\footnote{\url{https://github.com/vinicreis/zookeeper-kotlin/tree/main/buildSrc}} que contem a configuração de \textit{build} que pode ser compartilhada entre todos os módulos do projeto, como grupo, nome e versão do pacote, dependências compartilhadas, configuração de \textit{plugins} em comum, entre outras configurações. A fim de centralizar as dependências em comum ao longo de todos os módulos, o Gradle também oferece uma funcionalidade chamada de catálogo de versões. Desta forma, todas as dependências são declaradas em um único arquivo, localizada no caminho \textit{/gradle/libs.versions.toml}. Neste arquivo estão declaradas todas as dependências do projeto e suas versões.

\subsection{Importação do plugin detekt nas configurações do Gradle}

Como a implementação do linter \textit{detekt} deve ser feita como um \textit{plugin} para todos os projetos, então, iniciamos o importando pelo Gradle ao módulo \textit{buildSrc} e adicionando o detekt no catálogo de versões, conforme a \Cref{fig:detekt_adding_on_version_catalog}.

\begin{figure}[H]
    \begin{lstlisting}[language=toml,numbers = none]
[versions]
...
detekt = "1.23.3"

[libraries]
...
detekt = { module = "io.gitlab.arturbosch.detekt:detekt-gradle-plugin", version.ref = "detekt" }
    \end{lstlisting}
    \caption{Adicionando o \textit{detekt} no catálogo de versões}
    \label{fig:detekt_adding_on_version_catalog}
\end{figure}

Note que fizemos sua declaração no formato de dependência ao invés de \textit{plugin}, pois estamos utilizando \textit{scripts} pré-compilados para compartilhar a lógica de \textit{build} entre módulos e este formato ainda não suporta a importação no formato de \textit{plugins} via catálogo de versões.

Com isso, podemos importar o \textit{plugin} no arquivo de \textit{build} na raiz do módulo \textit{buildSrc} através do catálogo de versões, exibido na \Cref{fig:add_plugin_as_dependency}.

\begin{figure}[H]
    \begin{lstlisting}[language=Kotlin,numbers = none]
dependencies {
    ...
    implementation(libs.detekt)
}
    \end{lstlisting}   
    \caption{Adicionando o plugin \textit{detekt} nas dependências do projeto de lógica de \textit{build}}
    \label{fig:add_plugin_as_dependency}
\end{figure}

Então, criamos o \textit{script} \textit{zookeeper.detekt.gradle.kts}, que conterá todas as configurações do \textit{detekt} para cada módulo que o importar. 

No primeiro bloco do arquivo, precisamos declarar quais \textit{plugins} serão importados para este \textit{script}. Veja na \Cref{fig:add_plugin_script}, que como o \textit{detekt} já foi declarado como dependência do projeto, basta declaramos seu identificador no bloco de plugins para importá-lo.

\begin{figure}[H]
    \begin{lstlisting}[language=Kotlin,numbers = none]
plugins {
    id("io.gitlab.arturbosch.detekt")
}
    \end{lstlisting}
    \caption{Importando o plugin no script de configuração do \textit{detekt}}
    \label{fig:add_plugin_script}
\end{figure}

Em seguida, declaramos nossas variáveis principais necessárias para configuração, como diretório do projeto, caminho dos arquivos de configuração, arquivo de configuração da análise de \textit{code smell} - que seria uma análise de possíveis vulnerabilidades mais profundas no funcionamento do código \cite{coodesh_code_smell}, arquivos fonte que devem ser avaliados e arquivos de artefatos e recursos (\textit{resources}) que devem ser ignorados.

\begin{figure}[H]
    \begin{lstlisting}[language=Kotlin,numbers = none]
val projectSource = file(projectDir)
val configFile = files("$rootDir/config/detekt.yml")
val baselineFile = file("$rootDir/config/baseline.xml")
val kotlinFiles = "**/*.kt"
val resourceFiles = "**/resources/**"
val buildFiles = "**/build/**"
    \end{lstlisting}
    \caption{Definição das variáveis de configuração do \textit{detekt}}
    \label{fig:add_config_variables_on_script}
\end{figure}

Feito isso, através da importação do plugin do \textit{detekt}, a IDE já nos dá acesso a DSL (\textit{domain specific language} - linguagem específica de domínio) disposta pelo \textit{detekt}, para configuração do próprio \textit{plugin}, das tarefas de execução da análise, entre outros. Neste primeiro bloco, acessamos a função da DSL de configuração do \textit{plugin}:

\begin{figure}[H]
    \begin{lstlisting}[language=Kotlin,numbers = none]
detekt {
    // Define se seguimos a configuracao como base e aplicar apenas as que definirmos explicitamente
    buildUponDefaultConfig = true
    // Define se todas as regras serao habilitadas por padrao
    allRules = false
    // Define o diretorio do codigo fonte a ser analisado
    source.setFrom(projectSource)
    // Define o diretorio de configuracoes do plugin
    config.setFrom(configFile)
    // Define o arquivo de baseline utilizado para a analise de Code Smell
    baseline = baselineFile
}
    \end{lstlisting}
    \caption{Definindo os valores de configuração do plugin \textit{detekt}}
\end{figure}

Assim que importamos o \textit{plugin} Gradle do \textit{detekt}, uma nova tarefa Gradle é criada, que será automaticamente atrelada a outras tarefas como de \textit{build}, teste, verificação, etc., para garantir que o projeto segue todas as configurações especificadas pelo \textit{plugin}. Porém, precisamos atribuir a esta tarefa algumas configurações:

\begin{figure}[H]
    \begin{lstlisting}[language=Kotlin,numbers = none]
tasks.withType<Detekt>().configureEach {
    // Define se a tarefa continua executando em caso de falha (mantendo seu resultado)
    gradle.startParameter.isContinueOnFailure = true

    // Inclui os arquivos em Kotlin na analise
    include(kotlinFiles)
    // Exclui os arquivos de resources e build na analise
    exclude(resourceFiles, buildFiles)

    // Configura quais relatorios devem ser gerados:
    reports {
        // Habilita a geracao do relatorio em HTML
        html.required.set(true)
        // e configura o caminho onde o arquivo sera salvo
        html.outputLocation.set(file("build/reports/detekt.html"))
        // Habilita a geracao do relatorio em Markdown
        md.required.set(true)
        // e configura o caminho onde o arquivo sera salvo
        md.outputLocation.set(file("build/reports/detekt.md"))
    }
}
    \end{lstlisting}
    \caption{Definindo as configurações das \textit{tasks} do Gradle}
    \label{fig:set_detekt_tasks_config}
\end{figure}

Note que precisamos definir que esta tarefa irá continuar em caso de falha mantendo seu resultado, para termos a análise de todos os módulos. Ainda que a análise de um deles falhe, queremos que todos os relatórios sejam gerados, para que o desenvolvedor que tenta submeter seu código tenha a oportunidade de sanar todas as inconsistências após apenas uma execução.

Por fim, resta apenas configuramos a versão da JVM (\textit{Java Virtual Machine}) onde o \textit{plugin} será executado.

\begin{figure}[H]
    \begin{lstlisting}[language=Kotlin,numbers = none,numbers = none]
tasks.withType<Detekt>().configureEach {
    jvmTarget = "17"
}

tasks.withType<DetektCreateBaselineTask>().configureEach {
    jvmTarget = "17"
}
    \end{lstlisting}
    \caption{Setando a versão da JVM para as \textit{tasks} do \textit{detekt}}
    \label{fig:set_jvm_version}
\end{figure}

Após a criação deste \textit{script}, o Gradle irá compilá-lo durante a sincronização e a utilização deste poderá feita adicionando-o como \textit{plugin} em cada módulo onde for necessário. Para isso, basta adicionar o trecho abaixo no bloco de \textit{plugins} do Gradle em cada um dos módulos cabíveis:

\begin{figure}[H]
    \begin{lstlisting}[language=Kotlin,numbers = none]
plugins {
    id("zookeeper.detekt")
}
    \end{lstlisting}
    \caption{Importando a configuração do \textit{detekt} como \textit{plugin}}
    \label{fig:importing_detekt_config}
\end{figure}

Note que o identificador que importamos se trata do nome do \textit{script} antes do trecho \textit{.gradle.kts} no módulo \textit{buildSrc}.

\subsection{Geração e configuração das regras de lint}

O \textit{detekt} possui a funcionalidade de gerar um arquivo de configuração preenchido com os valores padrão da qual podemos usar como base para realizar os ajustes necessários. Para isso, baste executarmos o comando abaixo na raiz do projeto:

\begin{figure}[H]
    \begin{lstlisting}[language=Bash,numbers = none]
./gradlew detektGenerateConfig
    \end{lstlisting}
    \caption{Executando a geração dos arquivos de configuração do \textit{detekt}}
    \label{fig:generating_detekt_config}
\end{figure}

Feito isso, temos o arquivo \textit{/config/detekt.yml} criado. Das regras que vimos no capítulo anterior, podemos configurá-las neste arquivo, para que elas sejam aplicadas a execução da análise do \textit{detekt}.

\subsection{Execução da verificação via detekt}

Após toda a configuração finalizada, basta executarmos o comando abaixo:

\begin{figure}[H]
    \begin{lstlisting}[language=Bash,numbers = none]
./gradlew detektMain
    \end{lstlisting}
    \caption{Executando verificação do \textit{detekt}}
    \label{fig:running_detekt}
\end{figure}

Este comando analisa todo o código-fonte existe no diretório \textit{**/src/main/} de todos os módulos, mostrando os possíveis resultados no console, de acordo com a configuração realizada. Note que, caso exista algum erro apontado pelo processo, além de exibir o tipo e o local do problema detectado, a execução da tarefa também retorna falha.

Como a tarefa de análise do \textit{detekt} se torna dependência de \textit{build} e executarmos os comandos Gradle \textit{check}, \textit{test} ou mesmo \textit{build}, o comando \textit{detekt} será executado. Então o processo pode ser facilmente incorporado em esteiras de CI/CD, verificando se o código-fonte está complacente com os padrões de projeto antes mesmo do processo de \textit{merge}.

\section{Resultados obtidos}

Com o projeto ZooKeeper migrado para Kotlin \footnote{Vide repositório \href{https://github.com/vinicreis/zookeeper-kotlin/tree/46fffa9bf96b8e13da71dfdf6a47ae5dc876c758}{vinicreis/zookeeper-kotlin no commit 46fffa9}}, executamos o processo de \textit{lint} (utilizando o \textit{detekt}) de acordo com a implementação de configuração detalhada na \Cref{sec:linter_implementation}.

Para demonstrar os resultados obtidos, detalhamos cada um dos alertas disparados, comparando com o código não conforme a cada regra apontado pela ferramenta (no commit \textit{46fffa9}) com o resultado final (no commit final - \textit{HEAD}).

\subsection{Condicional complexa (\textit{ComplexCondition})}

A regra de condicional complexa faz parte do conjunto de regras de complexidade e é definida por uma quantidade alta de verificações encadeadas para formar uma condicional que dispare determinado comportamento. A regra visa melhorar a legibilidade e entendimento do código, aconselhando o usuário a mover essas condições para funções ou variáveis nomeadas de forma clara, demonstrando o que a regra em questão representa\cite{detekt_complex_condition_rule}.

O trecho de código onde o \textit{linter} apontou a regra de condicional complexa pode ser visualizado na \Cref{fig:detekt_complex_condition_before_example}.

\begin{figure}[H]
    \centering
    \begin{lstlisting}[language=Kotlin]
class KeyValueRepository {
    ...
    fun find(key: String, timestamp: Long?): Entry? {
        val result = data.getOrDefault(key, null)

        if (timestamp != null && timestamp > 0 && result != null && result.timestamp < timestamp)
            throw OutdatedEntryException(key, result.timestamp)

        return result
    }
    ...
}
    \end{lstlisting}
    \caption{\textit{KeyValueRepository.kt} - Trecho não conforme com a regra de condicional complexa}
    \label{fig:detekt_complex_condition_before_example}
\end{figure}

Já na \Cref{fig:detekt_complex_condition_after_example} podemos observar a mesma lógica aplicada de forma a não violar a conformidade em relação à regra em questão.

\begin{figure}[H]
    \centering
    \begin{lstlisting}[language=Kotlin]
class KeyValueRepository {
    ...
    fun find(key: String, timestamp: Long?): Entry? {
        return data.getOrDefault(key, null)?.also { entry ->
            timestamp?.let {
                if(entry.timestamp < it) error("Outdated entry found!")
            } ?: entry
        }
    }
    ...
}
    \end{lstlisting}
    \caption{\textit{KeyValueRepository.kt} - Exemplo de não violação da regra de condicional complexa}
    \label{fig:detekt_complex_condition_after_example}
\end{figure}

\subsection{Injeção de \textit{dispatcher} de corrotinas (\textit{InjectDispatcher})}

A regra em questão pertence ao conjunto de regras de corrotinas e estabelece que os \textit{dispatchers} de corrotinas devem ser injetados nas classes através de seus construtores para facilitar os testes unitários. Isto, pois, é uma boa prática definida pelo Google utilizar \textit{dispatchers} de teste em testes unitários e de integração para tornar o teste mais determinístico\cite{android_corroutines_test_best_practices}.

Observe a definição inicial da classe onde o erro foi apontado na \Cref{fig:detekt_inject_dispatcher_before_example}.

\begin{figure}[H]
    \centering
    \begin{lstlisting}[language=Kotlin]
class ControllerImpl(
    override val port: Int,
    debug: Boolean
) : Controller {
    private val coroutineScope = CoroutineScope(SupervisorJob() + Dispatchers.IO)
    ...
    override fun start() {
        timestampJob = coroutineScope.launch { timestampRepository.run() }
        dispatcherJob = coroutineScope.launch { dispatcher.run() }
    }
    ...
}
    \end{lstlisting}
    \caption{\textit{ControllerImpl.kt} - Trecho não conforme com a regra de injeção de \textit{dispatcher} de corrotinas}
    \label{fig:detekt_inject_dispatcher_before_example}
\end{figure}

Já na \Cref{fig:detekt_inject_dispatcher_after_example}, note que o \textit{dispatcher} passa a ser injetado através do construtor da classe \textit{ControllerImpl}.

\begin{figure}[H]
    \centering
    \begin{lstlisting}[language=Kotlin]
class ControllerImpl(
    override val port: Int,
    debug: Boolean,
    private val coroutineScope: CoroutineScope
) : Controller {
    ...
    override fun start() {
        timestampJob = coroutineScope.launch { timestampRepository.run() }
        dispatcherJob = coroutineScope.launch { dispatcher.run() }
    }
    ...
}
    \end{lstlisting}
    \caption{\textit{ControllerImpl.kt} - Trecho conforme com a regra de injeção de \textit{dispatcher} de corrotinas}
    \label{fig:detekt_inject_dispatcher_after_example}
\end{figure}

\subsection{Captura de exceção genérica (\textit{TooGenericExceptionCaught})}

A regra de captura de exceção genérica pertence ao conjunto de regras de exceções. Ela reporta ocorrências onde blocos \textit{catch} esperam exceções de um tipo muito genérico. Por padrão, esta regra reporta a captura de exceções do tipo \textit{ArrayIndexOutOfBoundsException}, \textit{Error}, \textit{Exception}, \textit{IllegalMonitorStateException}, \textit{IndexOutOfBoundsException}, \textit{NullPointerException}, \textit{RuntimeException} e \textit{Throwable}.

A ideia desta verificação é recomendar a captura de exceções específicas e que estejam relacionadas ou sejam esperadas no escopo da classe em questão\cite{detekt_too_generic_exception_caught_rule}. Do contrário, exceções não esperadas podem ser capturadas, e, então, ou um comportamento não esperado é mascarado ou o bloco \textit{catch} pode falhar e/ou não tratar da melhor forma a exceção capturada.

Veja na \Cref{fig:detekt_too_generic_exception_before_example} um exemplo onde a regra foi disparada durante a análise.

\begin{figure}[H]
    \centering
    \begin{lstlisting}[language=Kotlin]
class ControllerImpl(
    override val port: Int,
    debug: Boolean,
    private val coroutineScope: CoroutineScope
) : Controller {
    ...
    override fun put(request: PutRequest): PutResponse {
        return try {
            ...
        } catch (e: Throwable) {
            ...
        }
    }
    ...
}
    \end{lstlisting}
    \caption{\textit{ControllerImpl.kt} - Exemplo de captura de exceção muito genérica na classe \textit{ControllerImpl}}
    \label{fig:detekt_too_generic_exception_before_example}
\end{figure}

Note que a exceção capturada neste caso era do tipo \textit{Throwable}, que é o tipo mais primitivo de exceção, isto é, da qual todas as exceções herdam. Observe um exemplo de código conforme a regra na \Cref{fig:detekt_too_generic_exception_after_example}.

\begin{figure}[H]
    \centering
    \begin{lstlisting}[language=Kotlin]
class ControllerImpl(
    override val port: Int,
    debug: Boolean,
    private val coroutineScope: CoroutineScope
) : Controller {
    ...
    override fun put(request: PutRequest): PutResponse {
        return try {
            ...
        } catch (e: IllegalStateException) {
            ...
        }
    }
    ...
}
    \end{lstlisting}
    \caption{\textit{ControllerImpl.kt} - Exemplo de captura de exceção específica na classe \textit{ControllerImpl}}
    \label{fig:detekt_too_generic_exception_after_example}
\end{figure}

\subsection{Declaração de classe e nome do arquivo incompatíveis (\textit{MatchingDeclarationName})}

De acordo com a convenção de código do Kotlin\cite{kotlin_code_conventions}, arquivos contendo apenas uma classe devem ser chamados com o nome da classe somado a sua extensão (``.kt``). Isto vale para todos os tipos de classes e interfaces. Se o arquivo contém múltiplas classes ou apenas declarações de alto nível, um nome que descreve o que arquivo contém deve ser utilizado. A nomenclatura do arquivo deve estar no formato \textit{upper camel case}, com a primeira letra maiúscula (\textit{Pascal case}).

Durante o desenvolvimento do projeto, foi criada a classe \textit{Result} para atender a todas as operações que retornavam um resultado. Vide exemplo na \Cref{fig:detekt_matching_declaration_name_before_example}.

\begin{figure}[H]
    \centering
    \begin{lstlisting}[language=Kotlin]
enum class Result {
    OK,
    NOT_FOUND,
    ERROR,
    TRY_AGAIN_ON_OTHER_SERVER;
}
    \end{lstlisting}
    \caption{\textit{Result.kt} - Exemplo de declaração de arquivo incompatível com a classe declarada}
    \label{fig:detekt_matching_declaration_name_before_example}
\end{figure}

Porém, esta classe foi renomeada para \textit{OperationResult} quando seu objetivo passou a ser representar os resultados retornados apenas em operações entre cliente e servidor. Por algum motivo, o arquivo não foi renomeado de acordo com o nome da classe e, por isso, a regra em questão foi disparada alertando que o código não segue a convenção do Kotlin.

Após o arquivo ser renomeado para ``OperationResult.kt``, mesmo nome da classe, o alerta não é mais disparado.

\subsection{Chamada insegura em um tipo anulável (\textit{UnsafeCallOnNullableType})}

Esta regra faz parte do conjunto de regras de potenciais \textit{bugs}.

Como já mencionado na \Cref{subsec:most_common_error_uninit_variable_kotlin}, quando utilizamos o operador \verb|!!| em Kotlin, estamos garantindo que uma variável anulável, no momento da chamada onde é utilizado, terá seu conteúdo devidamente preenchido. Porém, esta prática é considerada insegura, justamente pela quantidade de formas que a linguagem disponibiliza para lidar com variáveis anuláveis. Tendo isso em vista, o \textit{linter} nos alerta da utilização deste operador para que outras abordagens sejam utilizadas \cite{detekt_unsafe_call_on_nullable_rule}.

Eis um exemplo de onde este tipo de chamada era realizado no projeto na \Cref{fig:detekt_unsafe_null_call_on_nullable_before_example}

\begin{figure}[H]
    \centering
    \begin{lstlisting}[language=Kotlin]
class ClientImpl(
    private val port: Int,
    private val serverHost: String,
    private val serverPorts: List<Int>,
    debug: Boolean
) : Client {
    ...
    override fun get(key: String) {
        try {
            val serverPort = serverPorts.getAnyOrNull()!!
            ...
        } catch (e: IOException) {
            log.e("Failed to process GET request", e)
        }
    }

    private fun <T> List<T>.getAnyOrNull(): T? = when {
        isEmpty() -> null
        else -> get(kotlin.random.Random.nextInt(0, size))
    }
    ...
}
    \end{lstlisting}
    \caption{\textit{ClientImpl.kt} - Exemplo de chamada insegura num objeto anulável}
    \label{fig:detekt_unsafe_null_call_on_nullable_before_example}
\end{figure}

Podemos propor duas abordagens para remover este tipo de utilização da chamada insegura. Na primeira, podemos disparar um erro, indicando que a \textit{serverPorts} está vazia e nenhum elemento pode ser acessado. Veja um exemplo na \Cref{fig:detekt_unsafe_null_call_on_nullable_after_first_example}.

\begin{figure}[H]
    \centering
    \begin{lstlisting}[language=Kotlin]
class ClientImpl(
    private val port: Int,
    private val serverHost: String,
    private val serverPorts: List<Int>,
    debug: Boolean
) : Client {
    ...
    override fun get(key: String) {
        try {
            val serverPort = serverPorts.getAnyOrNull() ?: error("No server port found to send request")
            ...
        } catch (e: IOException) {
            log.e("Failed to process GET request", e)
        } catch (e: IllegalStateException) {
            log.e("No server port found to send requests", e)
        }
    }

    private fun <T> List<T>.getAnyOrNull(): T? = when {
        isEmpty() -> null
        else -> get(Random.nextInt(0, lastIndex))
    }
    ...
}
    \end{lstlisting}
    \caption{\textit{ClientImpl.kt} - Primeiro exemplo de tratamento pré-condição de variável nula utilizando \textit{error}}
    \label{fig:detekt_unsafe_null_call_on_nullable_after_first_example}
\end{figure}

Outra alternativa seria alterar a função de extensão \textit{getAnyOrNull} para que ela busque um dos índices válidos da lista, como podemos observar na \Cref{fig:detekt_unsafe_null_call_on_nullable_after_second_example}.

\begin{figure}[H]
    \centering
    \begin{lstlisting}[language=Kotlin]
class ClientImpl(
    private val port: Int,
    private val serverHost: String,
    private val serverPorts: List<Int>,
    debug: Boolean
) : Client {
    ...
    override fun get(key: String) {
        try {
            val serverPort = serverPorts.getAnyOrNull()
            ...
        } catch (e: IOException) {
            log.e("Failed to process GET request", e)
        } catch (e: IllegalStateException) {
            log.e("No server port found to send requests", e)
        }
    }

    private fun <T> List<T>.getAnyOrNull(): T = get(Random.nextInt(lastIndex))
    ...
}
    \end{lstlisting}
    \caption{\textit{ClientImpl.kt} - Segundo exemplo delegando o acesso ao elemento a função de extensão de \textit{List}}
    \label{fig:detekt_unsafe_null_call_on_nullable_after_second_example}
\end{figure}

\subsection{Quebra de linha ao final do arquivo (\textit{NewLineAtEndOfFile})}

Esta regra pertence ao grupo das regras de estilo \cite{detekt_new_line_at_end_of_file_rule} e alerta sobre a falta de uma quebra de linha ao final do arquivo, apenas para cumprir com a convenção de estilo do Kotlin.

\subsection{Utilize as funções \textit{check} ou \textit{error} (\textit{UseCheckOrError})}

Esta regra faz parte do conjunto de regras de estilo. Como o Kotlin dispõe formas concisas de verificar invariantes e condições (sejam elas necessárias para início de uma operação qualquer - pré-condições, ou requerimento ao fim de uma operação - pós-condições), não é necessário utilizar o lançamento de exceções comuns que indiquem um estado não esperado (por exemplo, lançamento de exceções do tipo \textit{IllegalStateException}).

Baseado no exemplo da \Cref{fig:detekt_unsafe_null_call_on_nullable_before_example}, tome como exemplo o trecho da \Cref{fig:detekt_use_error_or_check_before_example}.

\begin{figure}[H]
    \centering
    \begin{lstlisting}[language=Kotlin]
class ClientImpl(
    private val port: Int,
    private val serverHost: String,
    private val serverPorts: List<Int>,
    debug: Boolean
) : Client {
    ...
    override fun get(key: String) {
        try {
            val serverPort = serverPorts.getAnyOrNull() 
                ?: throw IllegalStateException("No server port found to send request")
            ...
        } catch (e: IOException) {
            log.e("Failed to process GET request", e)
        } catch (e: IllegalStateException) {
            log.e("No server port found to send requests", e)
        }
    }

    private fun <T> List<T>.getAnyOrNull(): T? = when {
        isEmpty() -> null
        else -> get(Random.nextInt(0, lastIndex))
    }
    ...
}
    \end{lstlisting}
    \caption{\textit{ClientImpl.kt} - Exemplo não conforme de lançamento ``manual`` de exceção para verificação de pré-condição}
    \label{fig:detekt_use_error_or_check_before_example}
\end{figure}

A proposta da regra é que podemos utilizar a função \textit{error} (vide exemplo na \Cref{fig:detekt_unsafe_null_call_on_nullable_after_first_example}) ou \textit{check} (vide exemplo na \Cref{fig:detekt_use_error_or_check_after_example}).

\begin{figure}[H]
    \centering
    \begin{lstlisting}[language=Kotlin]
class ClientImpl(
    private val port: Int,
    private val serverHost: String,
    private val serverPorts: List<Int>,
    debug: Boolean
) : Client {
    ...
    override fun get(key: String) {
        try {
            val serverPort = serverPorts.getAnyOrNull()
            
            checkNotNull(serverPort) { "No server port found to send request" }
            ...
        } catch (e: IOException) {
            log.e("Failed to process GET request", e)
        } catch (e: IllegalStateException) {
            log.e("No server port found to send requests", e)
        }
    }

    private fun <T> List<T>.getAnyOrNull(): T? = when {
        isEmpty() -> null
        else -> get(Random.nextInt(0, lastIndex))
    }
    ...
}
    \end{lstlisting}
    \caption{\textit{ClientImpl.kt} - Exemplo de tratamento de pré-condição utilizando \textit{checkNotNull}}
    \label{fig:detekt_use_error_or_check_after_example}
\end{figure}

\subsection{Utilização de \textit{imports} globais (\textit{WildcardImport})}

Esta regra pertence ao conjunto de regras de estilo. Ela reporta a importação de objetos externos ao arquivo em questão de forma global. Evitar este tipo de prática deixa mais claro quais objetos foram importados de outros arquivos, além de prevenir conflitos de nomenclatura.

No exemplo na \Cref{fig:detekt_wildcard_imports_before_example}, todas as classes, funções de alto-nível e variáveis publicas no pacote \textit{kotlinx.coroutines} se tornaram globais no contexto do arquivo de exemplo.

\begin{figure}[H]
    \centering
    \begin{lstlisting}[language=Kotlin]
import kotlinx.coroutines.*
    \end{lstlisting}
    \caption{\textit{Dispatcher.kt} - Exemplo \textit{import} global no contexto do arquivo}
    \label{fig:detekt_wildcard_imports_before_example}
\end{figure}

Logo, como a função \textit{withContext}, por exemplo, está definida no pacote \textit{kotlinx.coroutines}, não seria possível criar uma função com a mesma assinatura dentro deste arquivo. Também não estaria explícito que é deste pacote que esta função está sendo importada. Para isso, é recomendável relizar a importação totalmente qualificada (isto é, utilizando todo o caminho do pacote e importanto membro individualmente). Vide exemplo na \Cref{fig:detekt_wildcard_imports_after_example}

\begin{figure}[H]
    \centering
    \begin{lstlisting}[language=Kotlin]
import kotlinx.coroutines.withContext
import kotlinx.coroutines.Dispatchers
    \end{lstlisting}
    \caption{\textit{Dispatcher.kt} - Exemplo de \textit{imports} totalmente qualificados}
    \label{fig:detekt_wildcard_imports_after_example}
\end{figure}

\subsection{Classe abstrata desnecessária (\textit{UnnecessaryAbstractClass})}

Esta regra faz parte do conjunto de regras de estilo que valida se classes abstratas possuem algum membro concreto, isto é, alguma variável/método não abstratos que possuem algum valor/implementação já definidos na classe. Se alguma classe abstrata não satisfazer esta regra, ela pode ser uma interface.

Por exemplo, veja que a classe da \Cref{fig:detekt_unnecessary_abstract_class_before_example} possui apenas propriedades não inicializadas.

\begin{figure}[H]
    \centering
    \begin{lstlisting}[language=Kotlin]
abstract class Response(
    val result: OperationResult,
    val message: String?
)
    \end{lstlisting}
    \caption{\textit{Response.kt} - Exemplo de classe abstrata que não possui membros concretos}
    \label{fig:detekt_unnecessary_abstract_class_before_example}
\end{figure}

Note que nenhuma definição é perdida se a convertemos para uma interface, como na \Cref{fig:detekt_unnecessary_abstract_class_after_example}.

\begin{figure}[H]
    \centering
    \begin{lstlisting}[language=Kotlin]
interface Response {
    val result: OperationResult
    val message: String?
}
    \end{lstlisting}
    \caption{\textit{Response.kt} - Exemplo de interface equivalente a classe abstrata}
    \label{fig:detekt_unnecessary_abstract_class_after_example}
\end{figure}

\subsection{Exceção capturada não utilizada (\textit{SwallowedException})}

A regra de exceção ``engolida`` pertence ao conjunto de regras de exceções. Ela reporta as ocorrências onde uma exceção capturada num bloco catch não é nem utilizada, nem passada adiante (como a causa, por exemplo) em uma possível nova exceção lançada.

Observe o trecho na \Cref{fig:detekt_swallowed_exception_before_example}. Note que a exceção do tipo \textit{IllegalArgumentException} não é lançada numa nova exceção nem é lida para que seu valor seja utilizado.

\begin{figure}[H]
    \centering
    \begin{lstlisting}[language=Kotlin]
class Worker(private val client: Client, debug: Boolean) {
    ...
    fun run() {
        while (running) {
            try {
                ...
            } catch (e: IllegalArgumentException) {
                println("Opcao invalida! Tente novamente ou pressione Ctrl+D para finalizar.")
            }
        }
    }
}
    \end{lstlisting}
    \caption{\textit{Worker.kt} - Exemplo de exceção ignorada dentro do bloco \textit{catch}}
    \label{fig:detekt_swallowed_exception_before_example}
\end{figure}

Agora, veja a implementação na \Cref{fig:detekt_swallowed_exception_after_example}. Perceba que a exceção é registrada no \textit{log} como um erro.

\begin{figure}[H]
    \centering
    \begin{lstlisting}[language=Kotlin]
class Worker(private val client: Client, debug: Boolean) {
    ...
    fun run() {
        while (running) {
            try {
                ...
            } catch (e: IllegalArgumentException) {
                log.e("Invalid option entered!", e)
                println("Opcao invalida! Tente novamente ou pressione Ctrl+D para finalizar.")
            }
        }
    }
}
    \end{lstlisting}
    \caption{\textit{Worker.kt} - Exemplo de utilização de exceção capturada no bloco \textit{catch}}
    \label{fig:detekt_swallowed_exception_after_example}
\end{figure}

Utilizar a exceção de qualquer forma já é suficiente para que ocorrências desta regra não sejam reportadas.

\subsection{Números ``mágicos`` (\textit{MagicNumber})}

A regra em questão faz parte do conjunto de regras de estilo e é definida pela detecção de valores constantes definidos diretamente no código (comumente referidos como ``hard-coded``). A proposta da regra é optar por definir estes valores como constantes bem nomeadas, de modo a deixar claro a que determinado valor se refere.

Observe a classe demonstrada na \Cref{fig:detekt_magic_number_before_example}. Note que o valor padrão, caso nenhum seja lido no console, será passado como a porta do servidor \textit{Controller} é definido diretamente no código. Mas sem analisar a implementação com mais profundidade, não é trivial concluir o que o valor significa e como ele impacta o código.

\begin{figure}[H]
    \centering
    \begin{lstlisting}[language=Kotlin]
interface Controller {
    ...
    companion object {
        private const val TAG = "ControllerMain"

        @JvmStatic
        fun main(args: Array<String>) {
            try {
                ...
                val port = readIntWithDefault("Digite o valor da porta do servidor", 10097)
                ...
            } catch (e: Throwable) {
                handleException(TAG, "Failed start Controller...", e)
            }
        }
    }
}
    \end{lstlisting}
    \caption{\textit{Controller.kt} - Exemplo de utilização de \textit{magic number} (valores \textit{hard-coded})}
    \label{fig:detekt_magic_number_before_example}
\end{figure}

Tome como referência o trecho na \Cref{fig:detekt_magic_number_after_example}. Analisando a chamada da função \textit{readWithDefault}, fica mais fácil compreender que o segundo parâmetro é o valor padrão e que este se refere ao valor padrão da porta do servidor \textit{Controller}.

\begin{figure}[H]
    \centering
    \begin{lstlisting}[language=Kotlin]
interface Controller {
    ...
    companion object {
        private const val TAG = "ControllerMain"
        private const val DEFAULT_PORT = 10097

        @JvmStatic
        fun main(args: Array<String>) {
            try {
                ...
                val port = readIntWithDefault("Digite o valor da porta do servidor", DEFAULT_PORT)
                ...
            } catch (e: Throwable) {
                handleException(TAG, "Failed start Controller...", e)
            }
        }
    }
}
    \end{lstlisting}
    \caption{\textit{Controller.kt} - Exemplo de alteração de um \textit{magic number} para uma variável constante mais descritiva}
    \label{fig:detekt_magic_number_after_example}
\end{figure}

\subsection{Comprimento máximo de linha excedido (\textit{MaxLineLength})}

Esta regra está contida no conjunto de regras de estilo e reporta quando alguma linha de código excede o limite definido nas configurações (onde o padrão é 120 caracteres). A motivação desta regra é que linhas longas prejudicam a legibilidade do código em telas menores e torna o código mais uniforme.

A correção da regra é simples e apenas requer a quebra das linhas conforme a convenção da linguagem \cite{kotlin_code_conventions}.

\subsection{Variáveis mutáveis podem se tornar valores apenas de leitura (\textit{VarCouldBeVal})}

Alerta se variáveis ou propriedades de classes que são mutáveis não tiveram seu valor alterado, podendo, portanto, ser valores apenas de leitura. Esta regra está contida no conjunto de regras de estilo.

Observe que, na \Cref{fig:detekt_var_could_be_val_before_example}, a variável \textit{running} não tem o valor alterado. Isso pois a corrotina é suspensa quando a chamada \textit{serverSocket.accept()} é executada, e, caso a chamada não obtenha resposta, uma exceção do tipo \textit{SocketTimeoutException} é lançada. Além disso, se a corrotina for cancelada, uma exceção do tipo \textit{CancellationException} será lançada. Logo, não é necessário interromper a execução através da troca do valor da variável \textit{running}.

\begin{figure}[H]
    \centering
    \begin{lstlisting}[language=Kotlin]
class Dispatcher(private val server: Server) {
    ...
    private var running = true

    suspend fun run() = withContext(Dispatchers.IO) {
        try {
            while (running) {
                val socket = serverSocket.accept()
                ...
            }
        } catch (e: SocketException) {
            log.e("Socket closed!", e)
        } finally {
            serverSocket.close()
        }
    }

    companion object {
        private const val TAG = "DispatcherThread"
    }
}
    \end{lstlisting}
    \caption{\textit{Dispatcher.kt} - Exemplo de utilização de uma variável mutável que não tem seu valor alterado}
    \label{fig:detekt_var_could_be_val_before_example}
\end{figure}

Isso dito, o código pode ser modificado de tal modo que a variável \textit{running} seja um valor apenas de leitura definido como \textit{true}. Pois, veja que seu valor serve apenas para manter o bloco rodando enquanto o fluxo não for interrompido pelos casos citados anteriormente. Veja um exemplo na \Cref{fig:detekt_var_could_be_val_after_example}.

\begin{figure}[H]
    \centering
    \begin{lstlisting}[language=Kotlin]
class Dispatcher(private val server: Server) {
    ...
    private val running = true
    
    suspend fun run() = withContext(Dispatchers.IO) {
        try {
            while (running) {
                val socket = serverSocket.accept()
                ...
            }
        } catch (e: SocketException) {
            log.e("Socket closed!", e)
        } finally {
            serverSocket.close()
        }
    }

    companion object {
        private const val TAG = "DispatcherThread"
    }
}
    \end{lstlisting}
    \caption{\textit{Dispatcher.kt} - Exemplo da troca de variável mutável por valor apenas de leitura}
    \label{fig:detekt_var_could_be_val_after_example}
\end{figure}

Note que se o valor \textit{running} fosse substituído pelo valor \textit{true}, de forma ``hard-coded``, não violamos nenhum tipo de regra de \textit{lint}, uma vez que o valor \textit{true} não se refere a nenhum significado específico para o contexto das regras de negócio da aplicação. O valor serve apenas para manter o bloco \textit{while} executando indefinidamente.

Observe que muitos trechos não conforme com as regras configuradas pela tarefa de \textit{lint} ofereciam um risco claro ao funcionamento da aplicação. Durante o desenvolvimento, buscando atender a necessidade de negócio, fica claro o quanto é simples não adotar boas práticas de desenvolvimento, deixando de prever alguns cenários de erros e expondo a aplicação a vulnerabilidades.

Além do ganho imediado obtido com a execução de ferramentas de \textit{lint} detectando potenciais problemas que não impactam o funcionamento da aplicação, temos melhorias das quais mensurar seu benefício não é tão trivial. Como, por exemplo, regras de formatação e estilo. Assim como é intuitivo focarmos na implementação do comportamento esperado, ficam de lado alguns pontos estéticos que possuem um valor não observável a curto prazo. 

O risco imediato oferecido nestes casos não é alto. Mas quando pensamos na sustentação destes software ou emprego de novas funcionalidades, ter um código-fonte legível gera uma curva de aprendizado menor. Isso faz com que novos desenvolvedores tenham tanto mais facilidade em interagir com estas aplicações, quanto menor risco de gerar novos defeitos durante este processo.

Note como também não é complexo manter uma constância na execução deste tipo de análise. Como vimos anteriormente, após implementada, executá-la é um processo simples e pouco custoso, permitindo empregar este tipo de análise em esteiras integração e publicação contínua.

\chapter{Considerações finais}\label{cap:final_considerations}

Mediante uma revisão estruturada na literatura científica, em busca de um modelo consistente de classificação de defeitos de software, encontramos alguns trabalhos que utilizam a classificação ODC (veja o \Cref{cap:cientific_library_research}), que leva em consideração a causa raiz do defeito com uma boa referência no ponto de vista de desenvolvimento. Logo, ela se mostra útil para classificarmos as ocorrências encontrados em uma aplicação real.

Esta análise das ocorrências, que envolve a contagem, agrupamento e análise dos erros mais comuns conforme a classificação encontrada anteriormente, nos permite entender que os erros mais frequentes estão diretamente relacionados com o processo de desenvolvimento. Isto nos leva a analisá-las mais a fundo para entender se existem práticas que mitiguem a ocorrência destes erros. Feito isso, analisamos possíveis abordagens, estratégias de desenvolvimento e ferramentas que possam ajudar neste processo.

No caso da escolha da linguagem, consideramos Java e Kotlin para uma comparação prática de como o Kotlin permite facilmente sobrepor alguns comportamentos que oferecem justamente o risco de ocorrências dos erros mais comuns encontrados na aplicação de exemplo. Do ponto de vista de técnicas de programação, foram propostas técnicas de controle de fluxo que também pode ajudar a capturar e tratar da forma correta eventos de erro, fugindo do paradigma mais comum que seria o tratamento de exceções, em Java. Já do ponto de vista de ferramentas, os \textit{linters}, se empregados corretamente, podem colaborar ainda mais e de forma automatizada para diminuir a probabilidade de que erros não esperados ocorram na aplicação em produção.

Como prova de conceito, a ferramente de \textit{linter} \textit{detekt} foi implementada em uma aplicação em Kotlin de forma acoplada ao seu processo de verificação e \textit{build}. Inclusive, durante a etapa de migração deste projeto de Java para Kotlin, diversas melhorias foram feitas a partir de relatórios extraídos da ferramenta em questão.

Em trabalhos futuros, podemos ampliar a pesquisa e analisar se, através da mineração de repositórios públicos, os defeitos mais comuns encontrados corroboram com os resultados aqui obtidos. Podemos também analisar \textit{issues} e correções nestes repositórios, aplicando modelos de trabalhos anteriores e aplicar resultados nos processos aqui realizados. Em paralelo a isso, podemos demonstrar outros indicadores quem demonstrem ganhos reais com a aplicação das abordagens propostas neste trabalho. Desta forma, podemos consolidar os pontos aqui propostos e ampliarmos as possibilidades de mitigação dos erros encontrados.


% ----------------------------------------------------------
% ELEMENTOS PÓS-TEXTUAIS (Referências, Glossário, Apêndices)
% ----------------------------------------------------------
%\postextual

% Referências bibliográficas
\bibliography{bibliografia}
%\glossary

% Apêndices
% % ----------------------------------------------------------
% Apêndices
% ----------------------------------------------------------

% ---
% Inicia os apêndices
% ---
\begin{apendicesenv}

% Imprime uma página indicando o início dos apêndices
\partapendices

% ----------------------------------------------------------
\chapter{Development Activity Data}\label{cap:appendixA}
% ----------------------------------------------------------

In this appendix, we share the analytical data from contributions addressed previously in this work. The images for code contributions are provided by GitHub. The code coverage images were made using IntelliJ IDEA code coverage tool.

\begin{figure}[ht]
  \includegraphics[width=\linewidth]{figs/appendix/asniffer-dev.PNG}
  \caption{Annotation Sniffer GitHub Status - Contributed with 5 commits changed 181 lines of code}
  \label{fig:annotation_sniffer_dev}
\end{figure}

\begin{figure}[ht]
  \includegraphics[width=\linewidth]{figs/appendix/visualizer-dev.PNG}
  \caption{Annotation Visualizer GitHub Status - Contributed with 13 commits changed +500.000 lines of code}
  \label{fig:annotation_visualizer_dev}
\end{figure}

\begin{figure}[ht]
  \includegraphics[width=\linewidth]{figs/appendix/sniffer-web-api-dev.PNG}
  \caption{Annotation Sniffer Web APP GitHub Status - Contributed with 13 commits changed 97.522 lines of code}
  \label{fig:annotation_webapp_dev}
\end{figure}

\begin{figure}[ht]
  \includegraphics[width=\linewidth]{figs/appendix/coverage.PNG}
  \caption{Annotation Sniffer Web APP Coverage per package}
  \label{fig:annotation_webapp_coverage}
\end{figure}

\begin{figure}[ht]
  \includegraphics[width=\linewidth]{figs/appendix/coverage2.PNG}
  \caption{Annotation Sniffer Web APP Coverage - 84\% lines of code covered}
  \label{fig:annotation_webapp_coverage2}
\end{figure}

\begin{figure}[ht]
  \includegraphics[width=\linewidth]{figs/appendix/github_actions.PNG}
  \caption{GitHub Actions CI for Annotation Sniffer Web APP}
  \label{fig:github_actions}
\end{figure}

\begin{figure}[ht]
  \includegraphics[width=\linewidth]{figs/appendix/asniffer_web_app_visualizer.PNG}
  \caption{Annotation Sniffer Web App visualization in the Annotation Visualizer Plugin}
  \label{fig:asniffer_web_app_visualizer}
\end{figure}

% ----------------------------------------------------------
\chapter{Survey Results}\label{cap:appendixB}
% ----------------------------------------------------------

This part contains the questions and answers results from the survey presented in \Cref{sub:validation_and_verification}. 

\begin{figure}[ht]
  \includegraphics[width=\linewidth]{figs/appendix/1_image.PNG}
  \caption{What is the name of the analyzed project?}
  \label{fig:project_name}
\end{figure}

\begin{figure}[ht]
  \includegraphics[width=\linewidth]{figs/appendix/2_image.PNG}
  \caption{What is the URL generated by the plugin? (Link that appears in the integrated browser after the project analysis)}
  \label{fig:url_generated}
\end{figure}

\begin{figure}[ht]
  \includegraphics[width=\linewidth]{figs/appendix/3_image.PNG}
  \caption{Is there any annotation of org.springframework.* in the view?}
  \label{fig:springframework_annotation}
\end{figure}

\begin{figure}[ht]
  \includegraphics[width=\linewidth]{figs/appendix/4_image.PNG}
  \caption{Which annotation schema (annotation family) is MOST used?}
  \label{fig:more_used}
\end{figure}

\begin{figure}[ht]
  \includegraphics[width=\linewidth]{figs/appendix/5_image.PNG}
  \caption{Which annotation schema (annotation family) is the LESS used?}
  \label{fig:less_used}
\end{figure}

\begin{figure}[ht]
  \includegraphics[width=\linewidth]{figs/appendix/6_image.PNG}
  \caption{Which class has the most annotations from the org.junit schema? (if own)}
  \label{fig:class_junit}
\end{figure}

\begin{figure}[ht]
  \includegraphics[width=\linewidth]{figs/appendix/7_image.PNG}
  \caption{It's easy to use the plugin}
  \label{fig:easy_to_use}
\end{figure}

\begin{figure}[ht]
  \includegraphics[width=\linewidth]{figs/appendix/8_image.PNG}
  \caption{The visual interface is clean and pleasant}
  \label{fig:clean_interface}
\end{figure}

\begin{figure}[ht]
  \includegraphics[width=\linewidth]{figs/appendix/9_image.PNG}
  \caption{The plugin documentation is enough for my use}
  \label{fig:documentation}
\end{figure}

\begin{figure}[ht]
  \includegraphics[width=\linewidth]{figs/appendix/10_image.PNG}
  \caption{Installing the plugin was easy}
  \label{fig:install_easy}
\end{figure}

\begin{figure}[ht]
  \includegraphics[width=\linewidth]{figs/appendix/11_image.PNG}
  \caption{Having a browser integrated with the IDE pleases me}
  \label{fig:browser_int}
\end{figure}

\begin{figure}[ht]
  \includegraphics[width=\linewidth]{figs/appendix/12_image.PNG}
  \caption{My experience with the plugin was pleasant}
  \label{fig:experience}
\end{figure}

\begin{figure}[ht]
  \includegraphics[width=\linewidth]{figs/appendix/13_image.PNG}
  \caption{User feedback}
  \label{fig:user_feedback}
\end{figure}


\end{apendicesenv}
% ---



% Anexos
%\include{postextual/anexos}

% Índice remissivo (Consultar manual)
%\phantompart
%\printindex

\end{document}
