\chapter{Introdução}\label{cap:introduction}

% Introdução
A sociedade está cada vez mais cercada e dependente de dispositivos baseados em computadores que movem nossas vidas em muitos aspectos. Estes dispositivos são projetados de modo que possam executar uma grande quantidade de softwares com diversas finalidades e o processo de criação destas aplicações, sendo ou não de forma completamente manual, está sujeita a erros.

% Contextualização
Ao tratarmos de desenvolvimento de software precisamos ter em mente que, apesar de computadores e softwares envolverem uma significativa parcela de automatização, consistência e velocidade em processos, a criação das ferramentas e recursos que moldam estes sistemas são ainda um processo humano que muito dependente deste fator \cite{human_factor_on_software_engineering}. Sem contar que atualmente existe uma enorme quantidade de ferramentas e recursos que podemos utilizar para criar softwares, cada um com suas características, pontos fracos e fortes. Além disso, esta pilha de tecnologias pode ser combinada de diversas maneiras de modo em que a integração entre elas também possam causar potenciais problemas.

Isso implica que ter clareza apenas das regras de negócio que uma aplicação deve atender não significa que o software desenvolvido para tal finalidade irá se comportar da maneira esperada. Também precisamos estar a par do poder das tecnologias que dispomos para podermos fazer melhor uso delas e explorar cada vez mais e melhor suas vantagens.

Uma falha ou defeito de um componente refere-se a uma condição anormal do mesmo, causada por problemas de lógica do sistema (no caso de software). Já o erro é a diferença entre o comportamento desejado/especificado e o real\cite{sandhof2006defeitos}. Ao longo deste trabalho, como analisamos os eventos gerados por defeitos ou erros em softwares, nos referimos aos dois da mesma forma.

Logo, caso surjam defeitos ou erros nessas aplicações, seja qualquer a razão, é importante que tenhamos em mente que significativas perdas podem ocorrer \cite{prejuizos_com_softwares} \cite{facebook_lost_millions_during_outage}. Seja em função da parada de uma parte fundamental ou importante da operação de uma empresa, ainda que se trate de um software intangível, mas também do tempo de manutenção e suporte, por exemplo. A implementação de novas funcionalidades precisa também ser considerada quando estamos planejando como será a arquitetura de uma solução, uma vez que software precisará estar em constante manutenção.

%Objetivo e Resultados
O principal objetivo deste trabalho, através dos resultados aqui demonstrados, é expressar a importância da adoção de boas práticas de programação, além de como a escolha da linguagem de programação empregada numa aplicação real pode ser crucial para mitigar os riscos da ocorrência de defeitos em tempo de execução em softwares em ambiente produtivo.

Primeiramente, é importante que tenhamos uma forma de classificar erros de engenharia de software no geral, para podermos entender quais de seus tipos são mais comumente encontrados. Para isso, buscamos na literatura científica quais referências podemos utilizar para termos uma classificação eficiente, que compreenda grande parte de suas possibilidades.

A classificação encontrada na etapa anterior foi aplicada em ocorrências capturadas via softwares de monitoramento de erros e exceções em uma aplicação de médio/grande porte, em termos de usuários ativos. Os erros analisados partem de exceções em Kotlin ou Java capturadas de forma proposital, ou não, para registrar um evento de erro capturado em tempo de execução. Isso facilita o processo de análise se comparado com a análise de \textit{issues} em repositórios públicos ou privados ou da correção aplicada a elas, por exemplo. Pois, observando as exceções lançadas, podemos ter uma ideia clara de qual é o defeito do ponto de vista de desenvolvimento do software em questão.

Analisando os defeitos, traçamos um paralelo entre as ocorrências observadas e o tipo de má prática que podem proporcionar maior risco de ocorrência destes erros. Baseado nelas, propomos outras abordagens, de baixo custo, que mitigam estes riscos. Serão propostas também técnicas que podem atuar de forma ativa na análise do código fonte da aplicação de modo acoplado a esteira de publicação. Desta forma, inúmeros problemas podem ser evitados em tempo de compilação e publicação, antes mesmo da revisão do código por pares durante o andamento da esteira de integração.

Por fim, comparamos duas linguagens muito parecidas (até interoperáveis entre si), que são Java e Kotlin, para entendermos quais recursos e paradigmas temos ao nosso dispor para evitarmos os defeitos analisados anteriormente. Além disso, utilizando o Kotlin como exemplo, aplicamos uma ferramenta de análise de estática de código \textit{linter} numa aplicação real, para entendermos quais benefícios temos com este tipo de processo e como podemos fazer para implementá-lo.

%Extrutura do Texto
Este trabalho é dividido em 6 partes. No \Cref{cap:cientific_library_research}, utilizando os procedimentos para uma revisão sistemática da literatura, estruturamos o levantamento de trabalhos que tratam a classificação de defeitos de software. No \Cref{cap:classification_on_real_world_app}, analisamos os dados das exceções capturadas por uma ferramenta de monitoramento em uma aplicação real. Utilizando a taxonomia obtida no \Cref{cap:cientific_library_research}, classificamos os dados e podemos ver quais são os erros mais comuns capturados nos dados. Já no \Cref{cap:most_common_errors_analysis}, propomos outras abordagens em relação às práticas que podem levar aos erros que foram encontrados nos dados analisados. Além disso, no \Cref{cap:linters}, nos aprofundamos na definição de ferramentas de análise estáticas, chamadas comumente de \textit{linters}, o que elas oferecem e um passo a passo de sua aplicação numa aplicação real, convertida de Java para Kotlin e os resultados atingidos com a análise da ferramenta. Por fim, no \Cref{cap:final_considerations} temos as considerações finais, explorando algumas conclusões dos dados obtidos e expondo os próximos trabalhos complementares a este.
