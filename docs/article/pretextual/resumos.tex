% ---
% RESUMOS
% ---

% RESUMO em português
\begin{resumo}
Com a modernização da sociedade, seu funcionamento está cada vez mais vinculado a execução de inúmeros tipos e aplicações de softwares em todos seus segmentos. O processo de desenvolvimento destas aplicações está sujeito a falhas. Falhas que podem gerar problemas que afetem sua plena atividade. Por conta disso, é importante adotarmos as melhores práticas que garantam a plena e correta execução destas aplicações. O objetivo principal deste trabalho é analisar quais são os defeitos mais frequentes em softwares e se estão relacionados com a pilha de tecnologias utilizada para sua criação. Iniciamos com uma revisão estruturada da literatura científica, em busca de uma classificação consolidada de defeitos de software. Com esta classificação, utilizando uma base de ocorrências de defeitos capturados numa aplicação real amplamente publicada, a aplicamos e analisamos os defeitos mais comuns visando entender se existem práticas de desenvolvimento, dentro da própria pilha utilizada, que possam mitigar estas ocorrências. Exploramos também ferramentas externas ao código-fonte do software em questão que seja capaz de analisá-lo em busca de vulnerabilidades. Como exemplo, utilizamos uma ferramenta de \textit{lint} chamada \textit{detekt} e registramos o processo de implementação, execução e revisão dos resultados obtidos em outro projeto de exemplo.

\vspace{\onelineskip}
 
\noindent 
\textbf{Palavras-chaves}: Java. Kotlin. Android. engenharia de software. odc. classificação. defeito. bug. linter. lint. detekt.
\end{resumo}

% ABSTRACT in english
\begin{resumo}[Abstract]
 \begin{otherlanguage*}{english}

With the modernization of society, its operation is increasingly tied to the execution of countless types and applications of software in all its segments. The development process of these applications is prone to failures. Failures that might generate problems that affect its full functioning. Because of this, it is important that we adopt the best practices that guarantee the full and correct execution of these applications. The main goal of this work is to analyze which are the most frequent defects in software and whether they are related to the stack used to create it. We started with a systematic review of the scientific literature, searching for a consolidated classification of software defects. With this classification, using a database of crash occurrences captured in a widely published real app, we classified and analyze the most common defects aiming to understand whether there are development practices, within the own stack used, that can mitigate these occurrences. We also explored external tools from the software's source code that are capable of analyzing it in search of vulnerabilities. As an example, we used a lint tool called \textit{detekt} and documented the process of implementation, execution and review of the results in another example project.

\vspace{\onelineskip}
 
\noindent 
\textbf{Keywords}: Java. Kotlin. Android. software engeneering. odc. classification. bug. linter. lint. detekt.
 \end{otherlanguage*}
\end{resumo}